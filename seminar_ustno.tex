\documentclass{beamer}

\usepackage[slovene]{babel}
\usepackage{amsfonts,amssymb,amsmath}
\usepackage[utf8]{inputenc}
\usepackage[T1]{fontenc}
\usepackage{lmodern}

\usefonttheme{serif}
\usetheme{Warsaw}

\def\qed{$\hfill\Box$}   % konec dokaza
\newtheorem{izrek}{Izrek}
\newtheorem{trditev}{Trditev}
\newtheorem{posledica}{Posledica}
\newtheorem{lema}{Lema}
\newtheorem{definicija}{Definicija}
\newtheorem{cilj}{Cilj}
\newtheorem{opomba}{Opomba}
\newtheorem{primer}{Primer}
\newtheorem{zgled}{Zgled}
\newtheorem{dokaz}{Dokaz}
% ===================================================================

\begin{document}
\title{Izreka o konvergenci in posplošitev kvocientnega kriterija}
\author{Timotej Stibilj}
\institute[FMF]{Fakulteta za matematiko in fiziko \\
Oddelek za matematiko}
\date{30. marec 2021}

\begin{frame}
   \titlepage
\end{frame}
% ===================================================================
\section{Uvod}
% ===================================================================
\begin{frame}
    \begin{block}{Motivacija}
    Pri določanju konvergence vrst $\sum_{n = 1}^{\infty}{a_n}$ z nenegativnimi členi si pomagamo z znanimi kriteriji.
    Spomnimo se na kvocientni kriterij. \\
    Če obstaja limita:
    \[
       \lim_{n \to \infty} \frac{a_{n + 1}}{a_n} = r \text{,}
    \]
    potem velja: \\
         \begin{itemize}
             \item če $ r < 1 $ vrsta konvergira,
             \item če $r > 1$ vrsta divergira,
             \item pri $r = 1$ kriterij odpove.
         \end{itemize}
 
    Radi bi posplošili \emph{kvocientni kriterij}, da bo z njim možno določiti konvergenco večjega nabora vrst.  
    \end{block}
\end{frame}

% -------------------------------------------------------------------
\begin{frame}
    \begin{primer}{Primer vrste za $r = 1$}: \\
       $\sum_{n = 1}^{\infty}{\frac{1}{n^p}}$, $p > 0$
    \end{primer}
    
    \vspace{\baselineskip}
    \pause
    $D = \lim_{n \to \infty} \frac{\frac{1}{(n+1)^p}}{\frac{1}{n^p}}$
    \pause
    $= \lim_{n \to \infty} (\frac{n}{n + 1})^p$ 
    \pause
    $= 1$
    \pause
    
    \vspace{\baselineskip}
    Kvocientni kriterij torej odpove.
    Opomba: Pri analizi 1 smo dokazali, da vrsta konvergira natanko tedaj ko s > 1.
    % člene združujemo in pokažemo, da je zaporedje delnih vsot omejeno/neomejeno glede na s
\end{frame}
% -------------------------------------------------------------------
\begin{frame}
    \begin{cilj}
        Predstavili bomo vrsto ekvivalenc za konvergenco vrst. Najpomembnješa rezultata sta:
        \begin{itemize}
            \item za $ 1 < q \in {\mathbb{N}}$, $\sum_{n = 1}^{\infty}{q^na_{q^n}}$ konvergira $\iff \sum_{n = 1}^{\infty}{a_{n}}$ konvergira, \\
            \item posplosen kvocientni kriterij.
        \end{itemize} 
    \end{cilj}

    \pause
    \vspace{\baselineskip}

    
    \begin{opomba}
        Sledeči izreki veljajo za \textcolor{red}{padajoče} vrste z nenegativnimi členi.\\
        BŠS lahko za vsako dano vrsto z nenegativnimi členi predpostavimo, da je padajoča.

        \vspace{\baselineskip}
        Dokaz: Uporabimo izrek o preureditvi absolutno konvergentnih vrst.
        %To lahko storimo zaradi izreka o preureditvi absolutno konvergentih vrst.
        %Vrsta absolutno konvergira $\implies$ vsaka preurejena vrsta konvergira k isti vsoti.
    \end{opomba}
\end{frame}

% ===================================================================
\section{1. Izrek}
% ===================================================================

\begin{frame}
    \begin{izrek}
        Naj bo ${a_n}$ padajoče zaporedje nenegativnih realnih števil. \\
        
        \vspace{0.15cm}
        Vrsta $\sum_{n = 1}^{\infty}{a_n}$ konvergira natanko tedaj, ko za vsako 
        podzaporedje $\{n_k\}$ zaporedja $\{n\}$ naravnih števil, konvergira vrsta 
        $\sum_{k = 1}^{\infty}{(n_{k+1} - n_k)a_{n_{k}}}$

        \vspace{0.35cm}
        Še več, velja vrsta ekvivalentnih trditev:
        \begin{enumerate}
            \item $\sum_{n = 1}^{\infty}{a_n}$ konvergira
            \item za vsako podzaporedje $\{n_k\}$ zaporedja $\{n\}$ \\
            konvergira $\sum_{k = 1}^{\infty}{(n_{k+1} - n_k)a_{n_{k + 1}}}$
            \item $\forall M > 0$, če za podzaporedje $\{n_k\}$ velja:
            \[
                (n_{k+1} - n_k) <= M * (n_k - n_{k - 1}),  \forall \quad 1 < k \in {\mathbb{N}} 
            \]
            potem $\sum_{k = 1}^{\infty}{(n_{k+1} - n_k)a_{n_{k}}}$ konvergira.
            \item cetrta tocka
        \end{enumerate}
    \end{izrek}
    
\end{frame}
% -------------------------------------------------------------------
\begin{frame}
    \begin{dokaz}
        napisi dokaz, odvisno od časa, vsaj 1 -> 2, ali verigo implikacij 1,2,3,4,1
    \end{dokaz}
\end{frame}

% -------------------------------------------------------------------
\begin{frame}
    \begin{zgled}
        Alternativni dokaz divergence harmonične vrste -> z implikacije 1 v 2
    \end{zgled}
\end{frame}

% -------------------------------------------------------------------
\begin{frame}
    \begin{posledica}
        %posledica izreka 1
        Če obstaja podzaporedje $\{n_k\}$, da velja:
        \[
            \lim_{k \to \infty} (n_{k+1} - n_k)a_{n{k+1}} \neq 0 
        \] 
        sledi divergenca za $\sum_{n = 1}^{\infty}{a_n}$
    \end{posledica}

    \vspace{0.45cm}
    \pause
    \begin{dokaz}
        Direktna posledica pogoja za konvergenco in implikacije $1 \implies 2$.
    \end{dokaz}

    
\end{frame}

% ===================================================================
\end{document}
