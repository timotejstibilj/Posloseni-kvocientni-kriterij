\documentclass{beamer}

\usepackage[slovene]{babel}
\usepackage{amsfonts,amssymb,amsmath}
\usepackage[utf8]{inputenc}
\usepackage[T1]{fontenc}
\usepackage{lmodern}

\usefonttheme{serif}
\usetheme{Warsaw}

\def\qed{$\hfill\Box$}   % konec dokaza
\newtheorem{izrek}{Izrek}
\newtheorem{trditev}{Trditev}
\newtheorem{posledica}{Posledica}
\newtheorem{lema}{Lema}
\newtheorem{definicija}{Definicija}
\newtheorem{cilj}{Cilj}
\newtheorem{opomba}{Opomba}
\newtheorem{primer}{Primer}
\newtheorem{zgled}{Zgled}
\newtheorem{dokaz}{Dokaz}
% ===================================================================

\begin{document}
\title{Izreka o konvergenci in posplošitev kvocientnega kriterija}
\author{Timotej Stibilj}
\institute[FMF]{Fakulteta za matematiko in fiziko \\
Oddelek za matematiko}
\date{30. marec 2021}

\begin{frame}
   \titlepage
\end{frame}
% ===================================================================
\section{Uvod}
% ===================================================================
\begin{frame}
    \begin{block}{Motivacija}
    Pri določanju konvergence vrst $\sum_{n = 1}^{\infty}{a_n}$ z nenegativnimi členi si pomagamo z znanimi kriteriji.
    Spomnimo se na kvocientni kriterij. \\
    Če obstaja limita:
    \[
       \lim_{n \to \infty} \frac{a_{n + 1}}{a_n} = r \text{,}
    \]
    potem velja: \\
         \begin{itemize}
             \item če $ r < 1 $ vrsta konvergira,
             \item če $r > 1$ vrsta divergira,
             \item pri $r = 1$ kriterij odpove.
         \end{itemize}
 
    Radi bi posplošili \emph{kvocientni kriterij}, da bo z njim možno določiti konvergenco večjega nabora vrst.  
    \end{block}
\end{frame}

% -------------------------------------------------------------------
\begin{frame}
    \begin{primer}{Primer vrste za $r = 1$}: \\
       $\sum_{n = 1}^{\infty}{\frac{1}{n^p}}$, $p > 0$
    \end{primer}
    
    \vspace{\baselineskip}
    \pause
    $D = \lim_{n \to \infty} \frac{\frac{1}{(n+1)^p}}{\frac{1}{n^p}}$
    $= \lim_{n \to \infty} (\frac{n}{n + 1})^p$ 
    $= 1$
    
    \vspace{\baselineskip}
    Kvocientni kriterij torej odpove.
    Opomba: Pri analizi 1 smo dokazali, da vrsta konvergira natanko tedaj ko s > 1.
    % člene združujemo in pokažemo, da je zaporedje delnih vsot omejeno/neomejeno glede na s
\end{frame}
% -------------------------------------------------------------------
\begin{frame}
    \begin{cilj}
        Predstavili bomo vrsto ekvivalenc za konvergenco vrst. Najpomembnješa rezultata sta:
        \begin{itemize}
            \item za $ 1 < q \in {\mathbb{N}}$, $\sum_{n = 1}^{\infty}{q^na_{q^n}}$ konvergira $\iff \sum_{n = 1}^{\infty}{a_{n}}$ konvergira, \\
            \item posplošen kvocientni kriterij.
        \end{itemize} 
    \end{cilj}

    \pause
    \vspace{\baselineskip}

    
    \begin{opomba}
        Sledeči izreki veljajo za \textcolor{red}{padajoče} vrste z nenegativnimi členi.\\
        BŠS lahko za vsako dano vrsto z nenegativnimi členi predpostavimo, da je padajoča.

        \vspace{\baselineskip}
        Dokaz: Izrek o preureditvi absolutno konvergentnih vrst.
        %To lahko storimo zaradi izreka o preureditvi absolutno konvergentih vrst.
        %Vrsta absolutno konvergira $\implies$ vsaka preurejena vrsta konvergira k isti vsoti.
    \end{opomba}
\end{frame}

% ===================================================================
\section{1. Izrek}
% ===================================================================

\begin{frame}
    \begin{izrek}
        Naj bo ${a_n}$ padajoče zaporedje nenegativnih realnih števil. \\
        
        \vspace{0.15cm}
        Vrsta $\sum_{n = 1}^{\infty}{a_n}$ konvergira natanko tedaj, ko za vsako 
        podzaporedje $\{n_k\}$ zaporedja $\{n\}$ naravnih števil, konvergira vrsta 
        $\sum_{k = 1}^{\infty}{(n_{k+1} - n_k)a_{n_{k}}}$

        \vspace{0.35cm}
        \pause
        Še več, velja vrsta ekvivalentnih trditev:
        \begin{enumerate}
            \item $\sum_{n = 1}^{\infty}{a_n}$ konvergira
            \item za vsako podzaporedje $\{n_k\}$ zaporedja $\{n\}$ \\
            konvergira $\sum_{k = 1}^{\infty}{(n_{k+1} - n_k)a_{n_{k + 1}}}$
            \item $\forall M > 0$, če za podzaporedje $\{n_k\}$ velja:
            \[
                (n_{k+1} - n_k) \leq M(n_k - n_{k - 1}),  \forall \quad 1 < k \in {\mathbb{N}} 
            \]
            potem $\sum_{k = 1}^{\infty}{(n_{k+1} - n_k)a_{n_{k}}}$ konvergira.
            \item cetrta tocka
        \end{enumerate}
    \end{izrek}
    
\end{frame}
% -------------------------------------------------------------------
\begin{frame}
    \begin{dokaz}
        napisi dokaz, odvisno od časa, vsaj 1 -> 2, ali verigo implikacij 1,2,3,4,1
    \end{dokaz}
\end{frame}

% -------------------------------------------------------------------
\begin{frame}
    \begin{zgled}
        Alternativni dokaz divergence harmonične vrste -> z implikacije 1 v 2
    \end{zgled}
\end{frame}

% -------------------------------------------------------------------
\begin{frame}
    \begin{posledica}
        %posledica izreka 1
        Če obstaja podzaporedje $\{n_k\}$, da velja:
        \[
            \lim_{k \to \infty} (n_{k+1} - n_k)a_{n{k+1}} \neq 0 
        \] 
        sledi divergenca za $\sum_{n = 1}^{\infty}{a_n}$
    \end{posledica}

    \vspace{0.45cm}
    \pause
    \begin{dokaz}
        Direktna posledica pogoja za konvergenco in implikacije $1 \implies 2$.
    \end{dokaz}   
\end{frame}
% -------------------------------------------------------------------

\begin{frame}
    % poseben primer za 4. točko 1. izreka je ta q^k
    \begin{izrek}
        Naj bo $\{a_n\}$ padajoče zaporedje in $ q \in {\mathbb{N}}$, $q > 1$. Tedaj je ekvivalentno:
        \begin{enumerate}
            \item $\sum_{n = 1}^{\infty}{a_n}$ konvergira,
            \item $\sum_{k = 1}^{\infty}{q^ka_{q^k}}$ konvergira,
            \item $\exists$ podzaporedje $\{n_k\}$ zaporedja $\{n\}$, da je $\{n_{k+1} - n_k\}$ 
            naraščajoče in konvergirata vrsti:\\
            $\sum_{k = 1}^{\infty}{\frac{1}{n_{k+1} - n_k}}$ in 
            $\sum_{k = 1}^{\infty}{(n_{k+1} - n_k)a_{n_k}}$  
            \item $\exists$ podzaporedje $\{n_k\}$ od $\{n\}$, da $\sum_{k = 1}^{\infty}{(n_{k+1} - n_k)a_{n_k}}$ konvergira.
        \end{enumerate}
    \end{izrek}    
\end{frame}

% -------------------------------------------------------------------
\begin{frame}{Dokaz}
    \begin{block}{}
        \begin{enumerate}   
            \item[(1)] $\sum_{n = 1}^{\infty}{a_n}$ konvergira $\implies$
            \item[(2)] $\sum_{k = 1}^{\infty}{q^ka_{q^k}}$ konvergira
        \end{enumerate}
    \end{block}     
    
    Naj bo $ 1 < q \in {\mathbb{N}}$.\\
    Uporabimo implikacijo $(1) \implies (3)$
    \pause
    iz izreka 1, za $n_k = q^k$:\\
    \[
        n_{k+1} - n_k \leq q(q^k - q^{k-1}) = q(n_{k} - n_{k - 1})
    \]
    \pause
    Sledi:
    \[
        \sum_{k = 1}^{\infty}{(n_{k+1} - n_k)a_{n_k}} = (q-1)\sum_{k = 1}^{\infty}{q^ka_{q^k}} 
        \quad \text{konvergira.}
    \]    
\end{frame}
% -------------------------------------------------------------------
\begin{frame}{Dokaz}
    \begin{block}{}
        \begin{enumerate} 
            \item[(2)] $\sum_{k = 1}^{\infty}{q^ka_{q^k}}$ konvergira $\implies$
            \item[(3)] $\exists$ podzaporedje $\{n_k\}$ zaporedja $\{n\}$, da je $\{n_{k+1} - n_k\}$ 
            naraščajoče in konvergirata vrsti:
            $\sum_{k = 1}^{\infty}{\frac{1}{n_{k+1} - n_k}}$ in 
            $\sum_{k = 1}^{\infty}{(n_{k+1} - n_k)a_{n_k}}$
        \end{enumerate}
    \end{block}     
    
    Vzemimo ${n_k} = q^k$.
    \pause 
    \begin{itemize}
        \item $\{n_{k+1} - n_k\}$ je naraščajoče, saj $ q > 1$
        \pause
        \item 
        $\sum_{k = 1}^{\infty}{\frac{1}{n_{k+1} - n_k}} = 
        \sum_{k = 1}^{\infty}{\frac{1}{q^{k + 1} - q^k}} =
        \frac{1}{q - 1}\sum_{k = 1}^{\infty}{(\frac{1}{q})^k}$
        \pause
        \item 
        $\sum_{k = 1}^{\infty}{(n_{k+1} - n_k)a_{n_k}} =
        \sum_{k = 1}^{\infty}{(q^{k + 1} - q^k)}a_{n_k} =
        (q - 1)\sum_{k = 1}^{\infty}{q^ka_{q^k}}$        
    \end{itemize}
    
\end{frame}
% -------------------------------------------------------------------

\begin{frame}{Dokaz}
    \begin{block}{}
        \begin{enumerate} 
            \item[(3)] $\exists$ podzaporedje $\{n_k\}$ zaporedja $\{n\}$, da je $\{n_{k+1} - n_k\}$ 
            naraščajoče in konvergirata vrsti:
            $\sum_{k = 1}^{\infty}{\frac{1}{n_{k+1} - n_k}}$ in 
            $\sum_{k = 1}^{\infty}{(n_{k+1} - n_k)a_{n_k}} \implies$
            \item[(4)] $\exists$ podzaporedje $\{n_k\}$ od $\{n\}$, da 
            $\sum_{k = 1}^{\infty}{(n_{k+1} - n_k)a_{n_k}}$ konvergira

        \end{enumerate}
    \end{block} 
    
    \vspace{0.3cm}
    \pause
    
    Očitno.\\
    Ostane nam še implikacija $ 1 \implies 4$.
    
\end{frame}

% -------------------------------------------------------------------

\begin{frame}{Dokaz}
    \begin{block}{}
        \begin{enumerate}  
            \item[(4)] $\exists$ podzaporedje $\{n_k\}$ od $\{n\}$, da 
            $\sum_{k = 1}^{\infty}{(n_{k+1} - n_k)a_{n_k}}$ konvergira $\implies$
            \item[(1)] $\sum_{n = 1}^{\infty}{a_n}$ konvergira.
        \end{enumerate}
    \end{block} 
    \pause
    
    Vzemimo tako podzaporedje  $\{n_k\}$, da $\sum_{n = 1}^{\infty}{a_n}$ konvergira.
    \vspace{0.2cm}

    %Označimo z $s_m$ zaporedje m-tih delnih vsot, t. j.
    $s_m := a_1 + a_2 + \cdots + a_m$


\end{frame}

% -------------------------------------------------------------------
\begin{frame}
    \begin{posledica}
        Naj bo $a_n$ padajoče zaporedje. Če velja:
        \[
            \lim_{n \to \infty}{q^ka_{q^k}} \neq 0 \text{,} 
        \]
        tedaj $\sum_{n = 1}^{\infty}{a_n}$ divergira.
    \end{posledica}
    \pause

    \begin{dokaz}
        Direktna posledica pogoja za konvergenco in implikacije $ (1) \implies (2)$
    \end{dokaz}
    
\end{frame}

% -------------------------------------------------------------------
\begin{frame}
    \begin{opomba}
        Pri analizi 1 (vaje) smo za padajoče zaporedje ${a_n}$ dokazali:
        \[
            \sum_{n = 1}^{\infty}{a_n} \quad \text{konvergira} \iff \text{konvergira vrsta}
            \sum_{n = 1}^{\infty}{2^ka_{2^k}}
        \]
        Naš izrek je posplošitev tega kriterija (Cauchyjevega kondenzacijskega kriterija) za $q = 2$.
    \end{opomba}
\end{frame}

% -------------------------------------------------------------------

\begin{frame}{Razširitev/Posplošitev ? kvocientnega kriterija}
    Po izreku 2, vemo: \\
    $\sum_{n = 1}^{\infty}{a_n}$ konvergira $\iff$
    $\sum_{k = 1}^{\infty}{q^ka_{q^k}}$, $q \in {\mathbb{N}}$, $q > 1$  konvergira.\\

    Zapišimo kvocienti kriterij za $\sum_{k = 1}^{\infty}{q^ka_{q^k}}$. Ta vrsta konvergira, če velja:

    \pause    
    \[  
        D := 
        \lim_{k \to \infty}{\frac{q^{k + 1}a_{q^{k + 1}}}{q^ka^k}}  =
        \lim_{k \to \infty}{\frac{qa_{q^{k + 1}}}{a^k}} < 1
        \iff
        \lim_{k \to \infty}{\frac{a_{q^{k + 1}}}{a^k}} < \frac{1}{q}
    \]
    
    Seveda velja tudi:\\
    če $\lim_{k \to \infty}{\frac{a_{q^{k + 1}}}{a^k}} > \frac{1}{q}$
    vrsta $\sum_{k = 1}^{\infty}{q^ka_{q^k}}$ divergira.    
\end{frame}
% -------------------------------------------------------------------
%zapisemo izrek, tako kot se za gre

\begin{frame}{Razširitev/Posplošitev ? kvocientnega kriterija}
    \begin{izrek}
        Naj bo $q \in {\mathbb{N}}$, $q > 1$ in ${a_n}$ zaporedje,
        ki zadošča $a_{n} \geq a_{n+1} \geq 0 \text{,} \quad \forall n \in {\mathbb{N}}$
        Tedaj velja:
        \pause

        \begin{enumerate}
            \item Če $\lim_{k \to \infty}{\frac{a_{q^{k + 1}}}{a^k}} < \frac{1}{q}$,
            potem vrsta $\sum_{n = 1}^{\infty}{a_n}$ konvergira.
            \item Če $\lim_{k \to \infty}{\frac{a_{q^{k + 1}}}{a^k}} > \frac{1}{q}$,
            potem vrsta $\sum_{n = 1}^{\infty}{a_n}$ divergira.
        \end{enumerate}
    \end{izrek}
    
    \pause
    \begin{dokaz}
        Uporaba izreka 2 in kvocientnega kriterija za $\sum_{k = 1}^{\infty}{q^ka_{q^k}}$.
    \end{dokaz}
\end{frame}
% -------------------------------------------------------------------
% Izrek je zares posplošitev kvocientnega kriterij:
\begin{frame}
    \begin{trditev}
        Naj bo $a_{n}$ padajoče zaporedje in naj obstaja limita
        $D := \lim_{n \to \infty}{\frac{a_{n + 1}}{a_n}}$. Tedaj za $q \in {\mathbb{N}}$, $q > 1$ velja:
        \[
            \lim_{k \to \infty}{\frac{a_{q^{k + 1}}}{a_{q^k}}} = 0
        \]

    \end{trditev}
    
\end{frame}
% -------------------------------------------------------------------

\begin{frame}
    \begin{dokaz}
    \end{dokaz}
\end{frame}

% -------------------------------------------------------------------
\begin{frame}
    Poglejmo si uporabo kriterija na dveh zgledih.
    \begin{zgled}
        Vemo, da za vrsto 
        $\sum_{n = 1}^{\infty}{\frac{1}{n^p}}$, $p > 0$
        kvocientni kriterij odpove.
        Poskusimo z razširjenim kvocientnim kriterijem.
        \[
            \lim_{k \to \infty}{\frac{a_{q^{k + 1}}}{a_{q^k}}}
        \]

    \end{zgled}
    \begin{zgled}
        mogoče se vrsta $\frac{1}{nln^p(n)}$
    \end{zgled}
\end{frame}

% ===================================================================
\end{document}

