\documentclass{beamer}

\usepackage[slovene]{babel}
\usepackage{amsfonts,amssymb,amsmath}
\usepackage[utf8]{inputenc}
\usepackage[T1]{fontenc}
\usepackage{lmodern}

\usefonttheme{serif}
\usetheme{Warsaw}

\def\qed{$\hfill\Box$}   % konec dokaza
\newtheorem{izrek}{Izrek}
\newtheorem{trditev}{Trditev}
\newtheorem{posledica}{Posledica}
\newtheorem{lema}{Lema}
\newtheorem{definicija}{Definicija}
\newtheorem{cilj}{Cilj}
\newtheorem{opomba}{Opomba}
\newtheorem{primer}{Primer}
\newtheorem{zgled}{Zgled}
\newtheorem{dokaz}{Dokaz}


\def\N{\mathbb{N}} % mnozica naravnih stevil
\def\Z{\mathbb{Z}} % mnozica celih stevil
\def\Q{\mathbb{Q}} % mnozica racionalnih stevil
\def\R{\mathbb{R}} % mnozica realnih stevil
\def\C{\mathbb{C}} % mnozica kompleksnih stevil
% ===================================================================

\begin{document}
\title{Posplošitev kvocientnega kriterija}
\author{Timotej Stibilj}
\institute[FMF]{Fakulteta za matematiko in fiziko \\
Oddelek za matematiko}
\date{9. april 2021}

\begin{frame}
   \titlepage
\end{frame}
% ===================================================================
\section{Uvod}
% ===================================================================
\begin{frame}
    \begin{block}{Motivacija}
    Pri ugotavljanju konvergence vrst $\sum_{n = 1}^{\infty}{a_n}$ z nenegativnimi členi si pomagamo z znanimi kriteriji.
    Spomnimo se na kvocientni kriterij. \\
    \pause
    Če obstaja limita:
    \[
       \lim_{n \to \infty} \frac{a_{n + 1}}{a_n} = D \text{,}
    \]
    potem velja: \\
         \begin{itemize}
             \item če $ D < 1 $ vrsta konvergira,
             \item če $ D > 1$ vrsta divergira,
             \item če $ D = 1$ o konvergenci ne moremo soditi.
         \end{itemize}
   
    \end{block}
\end{frame}

% -------------------------------------------------------------------
\begin{frame}
    \begin{primer}{Primer vrste za $D = 1$}: \\
       $\sum_{n = 1}^{\infty}{\frac{1}{n^p}}$, $p > 0$
    \end{primer}
    
    \vspace{\baselineskip}
    \pause
    $D = \lim_{n \to \infty} \frac{\frac{1}{(n+1)^p}}{\frac{1}{n^p}}$
    $= \lim_{n \to \infty} (\frac{n}{n + 1})^p$ 
    $= 1$
    \pause
    \vspace{\baselineskip}
    Kvocientni kriterij torej ne da odgovora.\\
    Opomba: Pri analizi 1 smo dokazali, da vrsta konvergira natanko tedaj ko s > 1.
    % člene združujemo in pokažemo, da je zaporedje delnih vsot omejeno/neomejeno glede na s
\end{frame}
% -------------------------------------------------------------------
\begin{frame}
    \begin{cilj}
        Radi bi posplošili \emph{kvocientni kriterij}, da bo z njim možno ugotoviti konvergenco večjega nabora vrst.
        Predstavili bomo vrsto ekvivalenc za konvergenco vrst. Najpomembnješa rezultata sta:
        \begin{itemize}
            \item za $ 1 < q \in {\mathbb{N}}$, $\sum_{n = 1}^{\infty}{q^na_{q^n}}$ konvergira $\iff \sum_{n = 1}^{\infty}{a_{n}}$ konvergira, \\
            \item posplošeni kvocientni kriterij.
        \end{itemize} 
    \end{cilj}
\end{frame}

\begin{frame}
    \begin{opomba}
        Sledeči izreki veljajo za vrste, kjer seštevamo člene \textcolor{red}{padajočega} zaporedja z nenegativnimi členi.\\
        Brez škode za splošnost lahko za vsako dano zaporedje z nenegativnimi členi predpostavimo, da je padajoče.
        
    \end{opomba}

    \pause
    \begin{izrek}
        Naj bo vrsta $\sum_{n = 1}^{\infty}{a_n}$ absolutno konvergentna.\\
        Tedaj za vsako bijekcijo:
        \[
            \pi: \N \rightarrow \N 
        \]
        tudi vrsta $\sum_{n = 1}^{\infty}{a_{\pi(n)}}$ konvergira in je 
        \[
            \sum_{n = 1}^{\infty}{a_{\pi(n)}} = \sum_{n = 1}^{\infty}{a_n}
        \]
    \end{izrek}
\end{frame}

\begin{frame}
    \begin{dokaz}[posledice]
        Pri vrstah z nenegativnimi členi je absolutna konvergenca ekvivalentna konvergenci.\\
        \pause
        Vzemimo tako bijekcijo $\pi: \N \rightarrow \N$, da je $\{a_{\pi(n)}\}$ padajoče.\\
        \pause
        Če pokažemo, da $\sum_{n = 1}^{\infty}{a_{\pi(n)}}$ divergira, potem sledi, da $\sum_{n = 1}^{\infty}{a_n}$ ne konvergira absolutno, torej divergira.\\
        \pause
        Če pokažemo, da $\sum_{n = 1}^{\infty}{a_{\pi(n)}}$ konvergira, vemo, da konvergira tudi absolutno in lahko uporabimo zgornji izrek za $\pi^{-1}$.
        \qed
    \end{dokaz}
    
\end{frame}

% ===================================================================
\section{1. Izrek}
% ===================================================================

\begin{frame}
    Pri določevanju konvergence vrste $\sum_{n = 1}^{\infty}{a_n}$, si lahko pomagamo s konvergenco vrste
    $\sum_{k = 1}^{\infty}{(n_{k+1} - n_k)a_{n_{k}}}$, kjer je $\{n_k\}$ podzaporedje zaporedja $\{n\}$ naravnih števil.    
\end{frame}

\begin{frame}
    \begin{izrek}[1]
        Naj bo $\{a_n\}$ padajoče zaporedje nenegativnih realnih števil. \\
        
        Tedaj so ekvivalentne naslednje izjave:
        \pause
        \begin{enumerate}
            \item Vrsta $\sum_{n = 1}^{\infty}{a_n}$ konvergira.
            \pause
            \item Za vsako podzaporedje $\{n_k\}$ zaporedja $\{n\}$ \\
            konvergira vrsta $\sum_{k = 1}^{\infty}{(n_{k+1} - n_k)a_{n_{k + 1}}}$.
            \pause
            \item Za vsak $M > 0$, če za podzaporedje $\{n_k\}$ zaporedja $\{a_n\}$ velja:\\
            $ (n_{k+1} - n_k) \leq M(n_k - n_{k - 1})$ za vsak $ 1 < k \in \N$,
            potem vrsta $\sum_{k = 1}^{\infty}{(n_{k+1} - n_k)a_{n_{k}}}$ konvergira.
            \pause
            \item Za vsak $M > 0$ obstaja podzaporedje $\{n_k\}$ zaporedja $\{n\}$, tako da velja:
            \[
                1 < n_{k+1} - n_k < (M+1)(n_k - n_{k - 1}),  \forall \quad 1 < k \in \N 
            \]
            in vrsta $\sum_{k = 1}^{\infty}{(n_{k+1} - n_k)a_{n_{k}}}$ konvergira.
        \end{enumerate}
    \end{izrek}
\end{frame}
% -------------------------------------------------------------------

\begin{frame}
    \begin{dokaz}
        Naj bo $\{a_n\}$ padajoče zaporedje in $a_n \geq 0$ za vsak $n \in \N$.\\
        $(1) \implies (2)$\\
        Denimo, da vrsta $\sum_{n = 1}^{\infty}{a_n}$ konvergira.
        \pause
        \[
        s_n = a_1 + a_2 + \ldots + a_n \text{,}\quad  n \in \N \quad \text{je omejeno}
        \] 
        \pause
        Za poljubno podzaporedje $\{n_k\}$ zaporedja naravnih števil $\{n\}$ označimo:
        \[
        t_m =  n_1a_{n_1} + (n_2 - n_1)a_{n_2} + \ldots + (n_{m + 1} - n_{m})a_{n_{m+ 1}} \quad m \in \N \text{.}
        \]

        \pause
        Zaporedje $\{t_m\}$ je naraščajoče.
        Pokažimo, da je omejeno.  
    \end{dokaz}    
\end{frame}

\begin{frame}
    \begin{dokaz}
        Za vsako število $n_{m + 1} \in \N$ obstaja $n \in \N$, da velja $n_{m + 1} < n$.
        \pause
        Ob predpostavki, da je zaporedje $\{a_n\}$ padajoče ocenimo:
        \[
            \begin{split}
                s_n & \geq \\
                & (a_1 + \ldots + a_{n_1}) + (a_{n_{1} + 1} + \ldots + a_{n_2}) + \ldots + (a_{n_{m} + 1} + \ldots + a_{n_{m+1}})\\
                & \geq n_1a_{n_1} + (n_2 - n_1)a_{n_2} + \ldots + (n_{m + 1} - n_{m})a_{n_{m+ 1}}\\
                & = t_m
            \end{split}
        \]
        \pause
        Zaporedje $\{t_m\}  = n_1a_1 + \sum_{k = 1}^{m}{(n_{k+1} - n_k)a_{n_{k + 1}}}$ je konvergentno, sledi
        da $\sum_{k = 1}^{\infty}{(n_{k+1} - n_k)a_{n_{k + 1}}}$ konvergira.
    \end{dokaz}
\end{frame}
\begin{frame}
    \begin{dokaz}
        $(2) \implies (3)$\\
        Denimo, da drži točka 2.
        Vzemimo poljuben $M \in \N$ in naj za podzaporedje $\{n_k\}$ zaporedja $\{n\}$ velja
        $ (n_{k+1} - n_k) \leq M(n_k - n_{k - 1})$, za vsak $ 1 < k \in \N$.
        \pause
        Sledi:
        \[
            (n_{k+1} - n_k)a_{n_k} \leq M(n_k - n_{k - 1})a_{n_k}, \quad \text{za vsak} \quad  1 < k \in \N \text{,}
        \]
        \pause
        Po predpostavki konvergira vrsta
        \[
            \begin{split}
            & \sum_{k = 1}^{\infty}{(n_{k+1} - n_k)a_{n_{k + 1}}}\\
            & = (n_2 -n_1)a_{n_2} + (n_3 - n_2)a_3 + \ldots \\
            & = \sum_{k = 2}^{\infty}{(n_k - n_{k - 1})a_{n_k}}
            \end{split}
        \]
    \end{dokaz}
\end{frame}
\begin{frame}
    \begin{dokaz}
        Sledi, da konvergira vrsta $\sum_{k = 2}^{\infty}{M(n_k - n_{k - 1})a_{n_k}}$.
        \pause
        Vrsta $\sum_{k = 1}^{\infty}{(n_{k+1} - n_k)a_{n_{k}}}$ konvergira po primerjalnem kriteriju.

        \qed
    \end{dokaz}
\end{frame}

% -------------------------------------------------------------------
\begin{frame}
    \begin{posledica}
        Naj bo ${a_n}$ padajoče zaporedje nenegativnih realnih števil.
        Če obstaja podzaporedje $\{n_k\}$ zaporedja \{n\} naravnih števil, da velja:
        \[
            \lim_{k \to \infty} (n_{k+1} - n_k)a_{n_{k+1}} \neq 0 
        \] 
        potem $\sum_{n = 1}^{\infty}{a_n}$ divergira.
    \end{posledica}
\end{frame}



% -------------------------------------------------------------------
\begin{frame}
    \begin{zgled}
        ( Alternativni dokaz divergence harmonične vrste)\\
        Za harmonično vrsto $\sum_{n = 1}^{\infty}{\frac{1}{n}}$ vzemimo $\{n_k\} = k!$. Sledi:
        \pause
        \[
            \sum_{k = 1}^{\infty}{(n_{k+1} - n_k)a_{n_{k + 1}}} =
            \sum_{k = 1}^{\infty}{((k + 1)! - k!) \frac{1}{(k+1)!}}=
            \sum_{k = 1}^{\infty}{\frac{k}{k + 1}}
        \]
        \pause
        Vrsta $\sum_{k = 1}^{\infty}{\frac{k}{k + 1}}$ divergira,
        saj $\lim_{k \to \infty}{\frac{k}{k + 1}} \neq 0$. \\
        \vspace{0.3 cm}
        Sledi $\sum_{n = 1}^{\infty}{\frac{1}{n}}$ divergira.
    \end{zgled}
\end{frame}

% -------------------------------------------------------------------

\begin{frame}
    % poseben primer za 4. točko 1. izreka je ta q^k
    \begin{izrek}[2]
        Naj bo $\{a_n\}$ padajoče zaporedje nenegativnih števil in $ q \in {\mathbb{N}}$, $q > 1$. Tedaj je ekvivalentno:
        \begin{enumerate}
            \item Vrsta $\sum_{n = 1}^{\infty}{a_n}$ konvergira.
            \pause
            \item Vrsta $\sum_{k = 1}^{\infty}{q^ka_{q^k}}$ konvergira.
            \pause
            \item Obstaja podzaporedje $\{n_k\}$ zaporedja $\{n\}$, da je $\{n_{k+1} - n_k\}$ 
            naraščajoče in konvergirata vrsti:\\
            $\sum_{k = 1}^{\infty}{\frac{1}{n_{k+1} - n_k}}$ in 
            $\sum_{k = 1}^{\infty}{(n_{k+1} - n_k)a_{n_k}}$. 
            \pause
            \item Obstaja podzaporedje $\{n_k\}$ od $\{n\}$, da vrsta $\sum_{k = 1}^{\infty}{(n_{k+1} - n_k)a_{n_k}}$ konvergira.
        \end{enumerate}
    \end{izrek}    
\end{frame}

% -------------------------------------------------------------------
\begin{frame}{Dokaz}
    \begin{block}{}
        \begin{enumerate}   
            \item[(1)] $\sum_{n = 1}^{\infty}{a_n}$ konvergira $\implies$
            \item[(2)] $\sum_{k = 1}^{\infty}{q^ka_{q^k}}$ konvergira
        \end{enumerate}
    \end{block}     
    
    Naj bo $ 1 < q \in {\mathbb{N}}$.\\
    Uporabimo implikacijo $(1) \implies (3)$
    iz izreka 1, za $n_k = q^k$:\\
    \[
        n_{k+1} - n_k \leq q(q^k - q^{k-1}) = q(n_{k} - n_{k - 1})
    \]
    \pause
    Sledi:
    \[
        \sum_{k = 1}^{\infty}{(n_{k+1} - n_k)a_{n_k}} = (q-1)\sum_{k = 1}^{\infty}{q^ka_{q^k}} 
        \quad \text{konvergira.}
    \]    
\end{frame}
% -------------------------------------------------------------------
\begin{frame}{Dokaz}
    \begin{block}{}
        \begin{enumerate} 
            \item[(2)] $\sum_{k = 1}^{\infty}{q^ka_{q^k}}$ konvergira $\implies$
            \item[(3)] Obstaja podzaporedje $\{n_k\}$ zaporedja $\{n\}$, da je $\{n_{k+1} - n_k\}$ 
            naraščajoče in konvergirata vrsti:
            $\sum_{k = 1}^{\infty}{\frac{1}{n_{k+1} - n_k}}$ in 
            $\sum_{k = 1}^{\infty}{(n_{k+1} - n_k)a_{n_k}}$
        \end{enumerate}
    \end{block}     
    
    Vzemimo ${n_k} = q^k$.
    \pause 
    \begin{itemize}
        \item $\{n_{k+1} - n_k\}$ je naraščajoče, saj $ q > 1$
        \pause
        \item 
        $\sum_{k = 1}^{\infty}{\frac{1}{n_{k+1} - n_k}} = 
        \sum_{k = 1}^{\infty}{\frac{1}{q^{k + 1} - q^k}} =
        \frac{1}{q - 1}\sum_{k = 1}^{\infty}{(\frac{1}{q})^k}$
        \pause
        \item 
        $\sum_{k = 1}^{\infty}{(n_{k+1} - n_k)a_{n_k}} =
        \sum_{k = 1}^{\infty}{(q^{k + 1} - q^k)}a_{n_k} =
        (q - 1)\sum_{k = 1}^{\infty}{q^ka_{q^k}}$        
    \end{itemize}
    
\end{frame}
% -------------------------------------------------------------------

\begin{frame}{Dokaz}
    \begin{block}{}
        \begin{enumerate} 
            \item[(3)] Obstaja podzaporedje $\{n_k\}$ zaporedja $\{n\}$, da je $\{n_{k+1} - n_k\}$ 
            naraščajoče in konvergirata vrsti:
            $\sum_{k = 1}^{\infty}{\frac{1}{n_{k+1} - n_k}}$ in 
            $\sum_{k = 1}^{\infty}{(n_{k+1} - n_k)a_{n_k}} \implies$
            \item[(4)] Obstaja podzaporedje $\{n_k\}$ od $\{n\}$, da 
            $\sum_{k = 1}^{\infty}{(n_{k+1} - n_k)a_{n_k}}$ konvergira.

        \end{enumerate}
    \end{block} 
    
    \vspace{0.3cm}
    \pause
    
    Očitno.\\
    
\end{frame}

% -------------------------------------------------------------------

\begin{frame}{Dokaz}
    \begin{block}{}
        \begin{enumerate}  
            \item[(4)] Obstaja podzaporedje $\{n_k\}$ od $\{n\}$, da 
            $\sum_{k = 1}^{\infty}{(n_{k+1} - n_k)a_{n_k}}$ konvergira $\implies$
            \item[(1)] $\sum_{n = 1}^{\infty}{a_n}$ konvergira.
        \end{enumerate}
    \end{block} 
    \pause
    
    Vzemimo tako podzaporedje  $\{n_k\}$, da $\sum_{n = 1}^{\infty}{a_n}$ konvergira in pokažimo, da je zaporedje delnih vsot
    vrste $\sum_{n = 1}^{\infty}{a_n}$ omejeno.
    \pause
    \[
    \begin{split}
        0 & < s_m := a_1 + a_2 + \cdots + a_m \\
        & \leq (a_1 + \ldots + a_{n_1}) + (a_{n_{1} + 1} + \ldots + a_{n_{2}})
        + \ldots  + (a_{n_{m} + 1} + \ldots + a_{n_{m+1}})\\
        & \leq n_{1}a_{1} + (n_2 - n_1)a_{n_{1} + 1} + \ldots + (n_{m+1} - n_m)a_{n_{m}+1}\\
        & \leq n_{1}a_{1} + (n_2 - n_1)a_{n_1} + \ldots + (n_{m+1} - n_{m})a_{n_{m}}\\
        & = n_{1}a_{1} + \sum_{k = 1}^{m}{(n_{k+1} - n_k)a_{n_k}}
    \end{split}
    \]

\end{frame}

% -------------------------------------------------------------------
\begin{frame}
    \begin{posledica}
        Naj bo $a_n$ padajoče zaporedje nenegativnih števil in $ q \in {\mathbb{N}}$, $q > 1$.. Če velja:
        \[
            \lim_{n \to \infty}{q^ka_{q^k}} \neq 0 \text{,} 
        \]
        tedaj $\sum_{n = 1}^{\infty}{a_n}$ divergira.
    \end{posledica}
    
\end{frame}

% -------------------------------------------------------------------
\begin{frame}
    \begin{opomba}
        Pri analizi 1 smo za padajoče zaporedje nenegativnih števil ${a_n}$ dokazali:
        \[
            \sum_{n = 1}^{\infty}{a_n} \quad \text{konvergira natanko tedaj, ko konvergira}
            \sum_{n = 1}^{\infty}{2^ka_{2^k}}
        \]
        Izrek 2 je posplošitev tega kriterija (Cauchyjevega kondenzacijskega kriterija), ki ga dobimo pri $q = 2$.
    \end{opomba}
\end{frame}

\begin{frame}{Zgled}
    Vzemimo vrsto $\sum_{n = 1}^{\infty}{\frac{1}{nln(n)}}$, za katero kvocientni kriterij ne da odgovora.
    \pause

    \vspace{0.3cm}
    \[
        \sum_{k = 1}^{\infty}{q^k\frac{1}{q^kln(q^k)}} =
        \sum_{k = 1}^{\infty}{\frac{1}{kln(q))}}
    \]

    Sledi divergenca.
    
\end{frame}

% -------------------------------------------------------------------
\begin{frame}
    \begin{izrek}[Posplošeni kvocientni kriterij]
        Naj bo $q \in \N$, $q > 1$ in $\{a_n\}$ zaporedje,
        ki zadošča $a_{n} \geq a_{n+1} \geq 0$, za vsak $n \in \N$.
        Tedaj velja:
        
        \begin{enumerate}
            \item Če $\lim_{k \to \infty}{\frac{a_{q^{k + 1}}}{a_{q^k}}} < \frac{1}{q}$,
            potem vrsta $\sum_{n = 1}^{\infty}{a_n}$ konvergira.
            \item Če $\lim_{k \to \infty}{\frac{a_{q^{k + 1}}}{a_{q^k}}} > \frac{1}{q}$,
            potem vrsta $\sum_{n = 1}^{\infty}{a_n}$ divergira.
        \end{enumerate}
    \end{izrek}
    
\end{frame}
% -------------------------------------------------------------------
\begin{frame}
    \begin{dokaz}
        Naj bo $\{a_n\}$ padajoče zaporedje nenegativnih števil in $ q \in {\mathbb{N}}$, $q > 1$.

        Denimo, da obstaja limita:
        \[ 
            \lim_{k \to \infty}{\frac{q^{k + 1}a_{q^{k + 1}}}{q^ka_{q^k}}}  =
            \lim_{k \to \infty}{\frac{qa_{q^{k + 1}}}{a_{q^k}}}  
            \text{.}
        \]
        \pause
        Tedaj velja:
        \begin{itemize}
            \item Če je $\lim_{k \to \infty}{\frac{qa_{q^{k + 1}}}{a_{q^k}}} < 1$, vrsta $\sum_{k = 1}^{\infty}{q^ka_{q^k}}$ konvergira.\\
            Če je $\lim_{k \to \infty}{\frac{a_{q^{k + 1}}}{a_{q^k}}} < \frac{1}{q}$, vrsta $\sum_{k = 1}^{\infty}{q^ka_{q^k}}$ konvergira.
            \item
            Če $\lim_{k \to \infty}{\frac{a_{q^{k + 1}}}{a_{q^k}}} > \frac{1}{q}$
            vrsta $\sum_{k = 1}^{\infty}{q^ka_{q^k}}$ divergira. 
        \end{itemize}
        \qed
    \end{dokaz}
\end{frame}

\begin{frame}
    \begin{trditev}
        Naj bo $\{a_{n}\}$ padajoče zaporedje in naj obstaja limita
        $D := \lim_{n \to \infty}{\frac{a_{n + 1}}{a_n}} < 1$. Tedaj za $q \in {\mathbb{N}}$, $q > 1$ velja:
        \[
            \lim_{k \to \infty}{\frac{a_{q^{k + 1}}}{a_{q^k}}} = 0
            \text{.}
        \]
    \end{trditev}
    
\end{frame}

% -------------------------------------------------------------------
\begin{frame}
    Vrnimo se k začetnemu primeru.
    \begin{zgled}
        Vemo, da za vrsto 
        $\sum_{n = 1}^{\infty}{\frac{1}{n^p}}$, $p > 0$
        kvocientni kriterij ne da odgovora.
        Poskusimo z razširjenim kvocientnim kriterijem.
        \pause
        \[
            \lim_{k \to \infty}{\frac{a_{q^{k + 1}}}{a_{q^k}}} =
            \lim_{k \to \infty}{\left( \frac{q^k}{q^{k+1}} \right)^p} = \frac{1}{q^p}
        \]
        \vspace{0.2cm}
        \pause
        Sledi, da vrsta $\sum_{n = 1}^{\infty}{\frac{1}{n^p}}$ konvergira za p > 1.

    \end{zgled}
    
\end{frame}

% ===================================================================
\end{document}

