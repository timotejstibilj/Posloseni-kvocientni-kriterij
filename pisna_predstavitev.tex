\documentclass[a4paper,12pt]{article}

\usepackage[slovene]{babel}
\usepackage{amsfonts,amssymb,amsmath}
\usepackage[utf8]{inputenc}
\usepackage[T1]{fontenc}
\usepackage{lmodern}



\def\N{\mathbb{N}} % mnozica naravnih stevil
\def\Z{\mathbb{Z}} % mnozica celih stevil
\def\Q{\mathbb{Q}} % mnozica racionalnih stevil
\def\R{\mathbb{R}} % mnozica realnih stevil
\def\C{\mathbb{C}} % mnozica kompleksnih stevil
\newcommand{\geslo}[2]{\noindent\textbf{#1} \quad \hangindent=1cm #2\\[-1pc]}


\def\qed{$\hfill\Box$}   % konec dokaza
\def\qedm{\qquad\Box}   % konec dokaza v matematičnem načinu
\newtheorem{izrek}{Izrek}
\newtheorem{trditev}{Trditev}
\newtheorem{posledica}{Posledica}
\newtheorem{lema}{Lema}
\newtheorem{pripomba}{Pripomba}
\newtheorem{definicija}{Definicija}
\newtheorem{zgled}{Zgled}

\title{Posplošitev kvocientnega kriterija za konvergenco vrste \\ 
\Large Seminar}
\author{Timotej Stibilj \\
Fakulteta za matematiko in fiziko \\
Oddelek za matematiko}
\date{21.\ marec 2021}

\begin{document}

\maketitle

\section{uvod}

\emph{Kvocientni kriterij}, imenovan tudi \emph{D'Alembertov kriterij} po francoskem matematiku Jeanu le Rond d'Alembertu
nam je pogosto v pomoč pri določevanju konvergence vrst $\sum_{n = 1}^{\infty}{a_n}$ z nenegativnimi členi. 
V poštev pride predvsem pri vrstah, ki vsebujejo n-te potence števil, npr. $\sum_{n = 1}^{\infty}{\frac{a^n}{n^s}}$, 
in pri vrstah, ki vsebujejo člen n faktorsko, npr. $\sum_{n = 1}^{\infty}{\frac{a^n}{n!}}$.
Spomnimo se, kako ga uporabimo:\\

\noindent
\begin{izrek}[Kvocientni kriterij]
    Če obstaja limita:
    \[
        D := \lim_{n \to \infty} \frac{a_{n + 1}}{a_n} \text{,}
    \]
    potem velja:
    \begin{itemize}
        \item če je $ D < 1 $ vrsta konvergira,
        \item če je $D > 1$ vrsta divergira,
        \item za $D = 1$ kriterij odpove.
    \end{itemize}
\end{izrek}

Poglejmo si preprost primer, pri katerem kvocientni kriterij odpove.

\begin{zgled}
    Vzemimo $a_n$ = $\frac{1}{n^p}$, $p \in \N$.
    \[
        D = \lim_{n \to \infty} \frac{a_{n + 1}}{a_n}
        = \lim_{n \to \infty} \frac{\frac{1}{(n+1)^p}}{\frac{1}{n^p}}
        = \lim_{n \to \infty} (\frac{n}{n + 1})^p
        = 1
    \]
    Limita $ \lim_{n \to \infty} \frac{a_{n + 1}}{a_n} $ je torej 1, od koder sledi, da si s kvocientnim kriterijem ne moremo pomagati.
\end{zgled}

Naravno se je vprašati, kako lahko kvocientni kriterij posplošimo, da bo zadostoval za določanje konvergence,
kjer kvocienti kriterij odpove, obenem tudi za prejšnji primer.\\
Možnih razširitev kriterija je več. V tem besedilu se bomo osredotočili na preproste kriterije, ki ne
zahtevajo poznavanja zahtevnih matematičnih konceptov. Predstavili bomo vrsto ekvivalentnih trditev
za konvergenco vrste $\sum_{n = 1}^{\infty}{a_n}$, kjer je ${a_n}$ padajoče zaporedje nenegativnih realnih števil, s preprostimi dokazi. Osrednja
izreka, ki ju želimo predstaviti sta:

\begin{itemize}
    \item Za $ 1 < q \in \N$, $\sum_{n = 1}^{\infty}{q^na_{q^n}}$ konvergira $\iff \sum_{n = 1}^{\infty}{a_{n}}$ konvergira.
    \item Posplošeni kvocientni kriterij, ki ga dobimo s pomočjo zgornjega izreka in običajnega kvocientnega kriterija.
    Le-tega lahko seveda, kakor ime pove, uporabimo v nekaterih primerih, v katerih običajen kvocientni kriterij odpove, 
    hkrati pa njegova uporaba, v smislu težavnosti računanja, ostaja enostavna.
\end{itemize}


Vsi izreki, ki so predstavljeni v nadaljevanju, veljajo za padajoče zaporedje nenegativnih števil.
Kaj lahko naredimo v primeru, ko imamo opravka z zaporedje $a_n$, ki ni padajoče?
Spomnimo se na izrek o preureditvi absolutno konvergentne vrste.

\begin{izrek}
    Naj bo vrsta $\sum_{n = 1}^{\infty}{a_n}$ absolutno konvergentna.\\
    Tedaj za vsako bijekcijo:
    \[\pi: \N \rightarrow \N \]
    tudi vrsta $\sum_{n = 1}^{\infty}{a_{\pi(n)}}$ konvergira in vsota vrst je enaka.
\end{izrek}

Torej lahko v primeru, ko se ukvarjamo z vsoto vrste $\sum_{n = 1}^{\infty}$, kjer je $\{a_n\}$ zaporedje z nenegativnimi členi, BŠS predpostavimo, 
da je $\{a_n\}$ padajoče. 


%%%%%%%%%%%%%%%%%%%%%%%%%%%%%%%%%%%%%%%%%%%%%%%%%%%%55

\begin{abstract}
\end{abstract}


\tableofcontents

\pagebreak


%%%%%%%%%%%%%%%%%%%%%%%%%%%%%%%%%%%%%%%%%%%%%%%%%%%%%%%%%%%%%%%%%%%%%%%%%%%%%%
%%%%%%%%%%%%%%%%%%%%%%%%%%%%%%%%%%%%%%%%%%%%%%%%%%%%%%%%%%%%%%%%%%%%%%%%%%%%%%





\bibliographystyle{siam}
\bibliographystyle{magic}

%%%%%%%%%%%%%%%%%%%%%%%%%%%%%%%%%%%%%%%%%%%%%%

\end{document}