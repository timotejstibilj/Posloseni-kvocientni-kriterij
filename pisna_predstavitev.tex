\documentclass[a4paper,12pt]{article}

\usepackage[slovene]{babel}
\usepackage{amsfonts,amssymb,amsmath}
\usepackage[utf8]{inputenc}
\usepackage[T1]{fontenc}
\usepackage{lmodern}



\def\N{\mathbb{N}} % mnozica naravnih stevil
\def\Z{\mathbb{Z}} % mnozica celih stevil
\def\Q{\mathbb{Q}} % mnozica racionalnih stevil
\def\R{\mathbb{R}} % mnozica realnih stevil
\def\C{\mathbb{C}} % mnozica kompleksnih stevil
\newcommand{\geslo}[2]{\noindent\textbf{#1} \quad \hangindent=1cm #2\\[-1pc]}


\def\qed{$\hfill\Box$}   % konec dokaza
\def\qedm{\qquad\Box}   % konec dokaza v matematičnem načinu
\newtheorem{izrek}{Izrek}
\newtheorem{trditev}{Trditev}
\newtheorem{posledica}{Posledica}
\newtheorem{lema}{Lema}
\newtheorem{opomba}{Opomba}
\newtheorem{definicija}{Definicija}
\newtheorem{zgled}{Zgled}

\title{Posplošitev kvocientnega kriterija za konvergenco vrste \\ 
\Large Seminar}
\author{Timotej Stibilj \\
Fakulteta za matematiko in fiziko \\
Oddelek za matematiko}
\date{9.\ april 2021}

\begin{document}

\maketitle

\section{uvod}

\emph{Kvocientni kriterij}, imenovan tudi \emph{D'Alembertov kriterij} po francoskem matematiku Jeanu le Rond d'Alembertu
nam je pogosto v pomoč pri ugotavljanju konvergence vrst $\sum_{n = 1}^{\infty}{a_n}$ z nenegativnimi členi. 
V poštev pride predvsem pri vrstah, ki vsebujejo $n$-te potence števil, npr. $\sum_{n = 1}^{\infty}{\frac{a^n}{n^s}}$, 
in pri vrstah, ki vsebujejo člen n faktorsko, npr. $\sum_{n = 1}^{\infty}{\frac{a^n}{n!}}$.
Spomnimo se, kako je formuliran:\\

\noindent
\begin{izrek}(Kvocientni kriterij) Naj bo $\sum_{n = 1}^{\infty}{a_n}$ vrsta s pozitivnimi členi.\\
    Če obstaja limita:
    \[
        D := \lim_{n \to \infty} \frac{a_{n + 1}}{a_n} \text{,}
    \]
    potem velja:
    \begin{itemize}
        \item če je $ D < 1 $, vrsta konvergira,
        \item če je $D > 1$, vrsta divergira,
        \item če je $D = 1$, o konvergenci v splošnem ne moremo soditi.
    \end{itemize}
\end{izrek}

Poglejmo si preprost primer, pri katerem kvocientni kriterij ne da odgovora.

\begin{zgled}
    Vzemimo $a_n$ = $\frac{1}{n^p}$, $p \in \N$.
    \[
        D = \lim_{n \to \infty} \frac{a_{n + 1}}{a_n}
        = \lim_{n \to \infty} \frac{\frac{1}{(n+1)^p}}{\frac{1}{n^p}}
        = \lim_{n \to \infty} \left( \frac{n}{n + 1} \right)^p
        = 1
    \]
    Limita $ \lim_{n \to \infty} \frac{a_{n + 1}}{a_n} $ je torej 1, od koder sledi, da si s kvocientnim kriterijem ne moremo pomagati.
\end{zgled}

Naravno se je vprašati, kako lahko kvocientni kriterij posplošimo, da bo zadostoval za ugotavljanje konvergence,
kjer kvocienti kriterij ne da odgovora, na primer za vrsto $\sum_{n = 1}^{\infty}{\frac{1}{n^p}}$.
S pomočjo izrekov, ki so predstavljeni v nadaljevanju, dobimo tudi alternativni dokaz znanega dejstva, da harmonična vrsta divergira.

Možnih razširitev kvocientnega kriterija je več. V tem besedilu se bomo osredotočili na preproste kriterije, ki ne
zahtevajo poznavanja zahtevnih matematičnih konceptov. Predstavili bomo vrsto ekvivalentnih trditev
za konvergenco vrste $\sum_{n = 1}^{\infty}{a_n}$, kjer je $\{a_n\}$ padajoče zaporedje nenegativnih realnih števil, s preprostimi dokazi. Osrednja
izreka, ki ju želimo predstaviti sta:

\begin{itemize}
    \item Za $ 1 < q \in \N$, $\sum_{n = 1}^{\infty}{q^na_{q^n}}$ konvergira $\iff \sum_{n = 1}^{\infty}{a_{n}}$ konvergira.
    \item Posplošeni kvocientni kriterij, ki ga dobimo s pomočjo zgornjega izreka in običajnega kvocientnega kriterija.
    Le-tega lahko seveda, kakor ime pove, uporabimo v nekaterih primerih, v katerih običajen kvocientni kriterij odpove, 
    hkrati pa njegova uporaba, v smislu težavnosti računanja, ostaja enostavna.
\end{itemize}


Vsi izreki, ki so predstavljeni v nadaljevanju, veljajo za padajoča zaporedja nenegativnih števil.
Kaj lahko naredimo v primeru, ko imamo opravka z zaporedjem $\{a_n\}$, ki ni padajoče?
Spomnimo se na izrek o preureditvi absolutno konvergentne vrste.

\begin{izrek}
    Naj bo vrsta $\sum_{n = 1}^{\infty}{a_n}$ absolutno konvergentna.\\
    Tedaj za vsako bijekcijo:
    \[\pi: \N \rightarrow \N \]
    tudi vrsta $\sum_{n = 1}^{\infty}{a_{\pi(n)}}$ konvergira in je 
    \[
        \sum_{n = 1}^{\infty}{a_{\pi(n)}} = \sum_{n = 1}^{\infty}{a_n}
    \]
\end{izrek}

Torej lahko v primeru, ko se ukvarjamo z vsoto vrste $\sum_{n = 1}^{\infty}{a_n}$, kjer je $\{a_n\}$ zaporedje z nenegativnimi členi, brez škode za splošnost predpostavimo, 
da je zaporedje $\{a_n\}$ padajoče. 

\section{Dva izreka o konvergenci vrst}
Pri ugotavljanju konvergence vrste $\sum_{n = 1}^{\infty}{a_n}$, si lahko pomagamo s konvergenco vrste
$\sum_{k = 1}^{\infty}{(n_{k+1} - n_k)a_{n_{k}}}$, kjer je $\{n_k\}$ podzaporedje zaporedja $\{n\}$ naravnih števil.
Formulirajmo izrek.

\begin{izrek}
    Naj bo $\{a_n\}$ padajoče zaporedje nenegativnih realnih števil. \\
    
    Tedaj so ekvivalentne naslednje izjave:
    \begin{enumerate}
        \item Vrsta $\sum_{n = 1}^{\infty}{a_n}$ konvergira.
        \item Za vsako podzaporedje $\{n_k\}$ zaporedja $\{n\}$ \\
        konvergira vrsta $\sum_{k = 1}^{\infty}{(n_{k+1} - n_k)a_{n_{k + 1}}}$.
        \item Za vsak $M > 0$, če za podzaporedje $\{n_k\}$ zaporedja $\{a_n\}$ velja:\\
        $ (n_{k+1} - n_k) \leq M(n_k - n_{k - 1}),  \text{za vsak} 1 < k \in \N$,
        potem vrsta $\sum_{k = 1}^{\infty}{(n_{k+1} - n_k)a_{n_{k}}}$ konvergira.
        \item Za vsak $M > 0$ obstaja podzaporedje $\{n_k\}$ zaporedja $\{n\}$, tako da velja:
        \[
            1 < n_{k+1} - n_k < (M+1)(n_k - n_{k - 1}),  \forall \quad 1 < k \in \N 
        \]
        in vrsta $\sum_{k = 1}^{\infty}{(n_{k+1} - n_k)a_{n_{k}}}$ konvergira.
    \end{enumerate}
\end{izrek}

\noindent
Dokazali bomo le implikaciji iz 1. v 2. in iz 2. v 3, ki ju bomo potrebovali v nadaljevanju. Celoten dokaz lahko najdete v \cite{convergence}.

\noindent
{\em Dokaz:\/}
Naj bo $\{a_n\}$ padajoče zaporedje in $a_n \geq 0$ za vsak $n \in \N$.\\
$(1) \implies (2)$\\
Denimo, da vrsta $\sum_{n = 1}^{\infty}{a_n}$ konvergira. Od tod sledi, da je zaporedje njenih delnih vsot:
\[
  s_n = a_1 + a_2 + \ldots + a_n \text{,}\quad  n \in \N
\] omejeno.
Za poljubno podzaporedje $\{n_k\}$ zaporedja naravnih števil $\{n\}$ označimo:
\[
  t_m =  n_1a_{n_1} + (n_2 - n_1)a_{n_2} + \ldots + (n_{m + 1} - n_{m})a_{n_{m+ 1}} \quad m \in \N \text{.}
\]

Za zaporedje $t_m$ vemo, da je naraščajoče, saj gre za vsoto čedalje več členov nenegativnega zaporedja $a_n$.
Želimo si dokazati še, da je omejeno, od koder bo sledilo, da zaporedje konvergira.  

Za vsako število $n_{m + 1} \in \N$ obstaja $n \in \N$, da velja $n_{m + 1} < n$.
Ob upoštevanju predpostavke, da je zaporedje $\{a_n\}$ padajoče naredimo oceno:
\[
    \begin{split}
        s_n & \geq (a_1 + \ldots + a_{n_1}) + (a_{n_{1} + 1} + \ldots + a_{n_2}) + \ldots + (a_{n_{m} + 1} + \ldots + a_{n_{m+1}})\\
        & \geq n_1a_{n_1} + (n_2 - n_1)a_{n_2} + \ldots + (n_{m + 1} - n_{m})a_{n_{m+ 1}}\\
        & = t_m \text{,}
    \end{split}
\]
pri čemer smo v prvi vrstici vzeli $n_{m + 1} < n$, v drugi vrstici pa smo upoštevali, da je zaporedje $\{a_n\}$ padajoče.

Iz ocene sledi, da je zaporedje $\{t_n\}$ omejeno, saj je omejeno zaporedje $\{s_n\}$ (sledi iz njegove konvergence).
Pokazali smo torej, da je $t_m  = n_1a_1 + \sum_{k = 1}^{m}{(n_{k+1} - n_k)a_{n_{k + 1}}}$ omejeno in naraščajoče, od koder sledi, da konvergira.
Od tod pa direktno sledi, da konvergira vrsta $\sum_{k = 1}^{\infty}{(n_{k+1} - n_k)a_{n_{k + 1}}}$, saj je 
$\sum_{k = 1}^{m}{(n_{k+1} - n_k)a_{n_{k + 1}}}$ ravno zaporedje njenih delnih vsot.

\noindent
$(2) \implies (3)$\\
Denimo, da drži točka 2.
Vzemimo poljuben $M \in \N$ in naj za podzaporedje $\{n_k\}$ zaporedja $\{n\}$ velja
$ (n_{k+1} - n_k) \leq M(n_k - n_{k - 1})$, za vsak $ 1 < k \in \N$.
Od tod očitno sledi:
\[
    (n_{k+1} - n_k)a_{n_k} \leq M(n_k - n_{k - 1})a_{n_k}, \quad \text{za vsak} \quad  1 < k \in \N \text{,}
\]
ker je zaporedje $\{a_n\}$ nenegativno.

Po predpostavki konvergira vrsta

\[
    \begin{split}
    & \sum_{k = 1}^{\infty}{(n_{k+1} - n_k)a_{n_{k + 1}}}\\
    & = (n_2 -n_1)a_{n_2} + (n_3 - n_2)a_3 + \ldots \\
    & = \sum_{k = 2}^{\infty}{(n_k - n_{k - 1})a_{n_k}} \text{.}
    \end{split}
\]
Sledi, da konvergira vrsta $\sum_{k = 2}^{\infty}{M(n_k - n_{k - 1})a_{n_k}}$ in po 
primerjalnem kriteriju konvergira tudi vrsta $\sum_{k = 1}^{\infty}{(n_{k+1} - n_k)a_{n_{k}}}$.

\qed

\begin{zgled}(Alternativni dokaz divergence harmonične vrste)\\
    Denimo, da imamo podzaporedje $\{n_k\}$ zaporedja naravnih števil $\{n\}$, 
    za katerega vrsta $\sum_{k = 1}^{\infty}{(n_{k+1} - n_k)a_{n_{k + 1}}}$ divergira. Potem po implikaciji
    iz 1. v 2. vrsta $\sum_{n = 1}^{\infty}{a_n}$ divergira.\\
    Za harmonično vrsto $\sum_{n = 1}^{\infty}{\frac{1}{n}}$ vzemimo $\{n_k\} = k!$. Sledi:
    \[
        \sum_{k = 1}^{\infty}{(n_{k+1} - n_k)a_{n_{k + 1}}} =
        \sum_{k = 1}^{\infty}{((k + 1)! - k!) \frac{1}{(k+1)!}}=
        \sum_{k = 1}^{\infty}{\frac{k}{k + 1}}
    \]
    Vrsta $\sum_{k = 1}^{\infty}{\frac{k}{k + 1}}$ divergira,
    saj $\lim_{k \to \infty}{\frac{k}{k + 1}} \neq 0$. Sledi, da harmonična vrsta divergira.
\end{zgled}


Iz ekvivalence med 1. in 2. iz izreka 3 in potrebnega pogoja za konvergenco vrste sledi naslednja posledica.
\begin{posledica}
    Naj bo ${a_n}$ padajoče zaporedje nenegativnih realnih števil.
    Če obstaja podzaporedje $\{n_k\}$ zaporedja \{n\} naravnih števil, da velja:
    \[
        \lim_{k \to \infty} (n_{k+1} - n_k)a_{n_{k+1}} \neq 0 
    \] 
    potem $\sum_{n = 1}^{\infty}{a_n}$ divergira.
\end{posledica}

Po izreku 3 velja: 
Konvergenca vrste $\sum_{n = 1}^{\infty}{a_n}$ implicira obstoj takega
podzaporedja $\{n_k\}$ od $\{n\}$, da $\sum_{k = 1}^{\infty}{(n_{k+1} - n_k)a_{n_{k}}}$ konvergira.
Obratno, če $\sum_{k = 1}^{\infty}{(n_{k+1} - n_k)a_{n_{k}}}$ konvergira, potem sledi, da 
$\sum_{n = 1}^{\infty}{a_n}$ konvergira.
Izkaže se, da lahko konvergenco $\sum_{n = 1}^{\infty}{a_n}$ popolnoma karakteriziramo
z $\sum_{k = 1}^{\infty}{q^ka_{q^k}}$, torej če za $\{n_k\}$ izberemo ${q^k}$, kjer je $q \in \N$, $q > 1$.

\begin{izrek}
    Naj bo $\{a_n\}$ padajoče zaporedje nenegativnih števil in $ q \in {\mathbb{N}}$, $q > 1$. Tedaj je ekvivalentno:
    \begin{enumerate}
        \item Vrsta $\sum_{n = 1}^{\infty}{a_n}$ konvergira.
        \item Vrsta $\sum_{k = 1}^{\infty}{q^ka_{q^k}}$ konvergira.
        \item Obstaja podzaporedje $\{n_k\}$ zaporedja $\{n\}$, da je $\{n_{k+1} - n_k\}$ 
        naraščajoče in konvergirata vrsti:\\
        $\sum_{k = 1}^{\infty}{\frac{1}{n_{k+1} - n_k}}$ in 
        $\sum_{k = 1}^{\infty}{(n_{k+1} - n_k)a_{n_k}}$. 
        \item Obstaja podzaporedje $\{n_k\}$ od $\{n\}$, da vrsta $\sum_{k = 1}^{\infty}{(n_{k+1} - n_k)a_{n_k}}$ konvergira.
    \end{enumerate}
\end{izrek}   

\noindent
{\em Dokaz:\/} \\Dokazali bomo verigo implikacij. $ (1)\implies (2)\implies (3) \implies (4)\implies(1)$.
$(1) \implies (2)$ Naj bo $ 1 < q \in \N$ in naj $\sum_{n = 1}^{\infty}{a_n}$ konvergira.\\
    Uporabimo implikacijo iz 1. v 3. iz izreka 3 za $n_k = q^k$:\\
    \[
        n_{k+1} - n_k \leq q(q^k - q^{k-1}) = q(n_{k} - n_{k - 1})
    \]
    Zaporedje $\{n_k\}$ torej zadošča pogoju $(n_{k+1} - n_k) \leq M(n_k - n_{k - 1})$,
    $\forall 1 < k \in \N$ za $M = q$.
    Sledi:
    \[
        \sum_{k = 1}^{\infty}{(n_{k+1} - n_k)a_{n_k}} = (q-1)\sum_{k = 1}^{\infty}{q^ka_{q^k}} 
        \quad \text{konvergira.}
    \]
    Od tod pa očitno sledi, da $\sum_{k = 1}^{\infty}{q^ka_{q^k}}$ konvergira.

\noindent
$(2) \implies (3)$ Vzemimo ${n_k} = q^k$.\\

\begin{itemize}
    \item $\{n_{k+1} - n_k\}$ je naraščajoče, saj $ q > 1$,
    \item Izračunajmo vrsto
    $\sum_{k = 1}^{\infty}{\frac{1}{n_{k+1} - n_k}} = 
    \sum_{k = 1}^{\infty}{\frac{1}{q^{k + 1} - q^k}} =
    \frac{1}{q - 1}\sum_{k = 1}^{\infty}{(\frac{1}{q})^k}$.\\
    To je geometrijska vrsta s koeficientom $\frac{1}{q}$ za $q > 1$. Torej je konvergentna.
    \item Izračunajmo še vrsto $\sum_{k = 1}^{\infty}{(n_{k+1} - n_k)a_{n_k}} =
    \sum_{k = 1}^{\infty}{(q^{k + 1} - q^k)}a_{n_k} =
    (q - 1)\sum_{k = 1}^{\infty}{q^ka_{q^k}}$.\\
    Iz predpostavke, da $\sum_{k = 1}^{\infty}{q^ka_{q^k}}$ konvergira, torej sledi,
    da $\sum_{k = 1}^{\infty}{(n_{k+1} - n_k)a_{n_k}}$ konvergira.
\end{itemize}


\noindent
Implikacija $(3) \implies (4)$ je očitna, ker za $\{n_k\}$ vzamemo isto zaporedje kot v točki 3.

\noindent
$(4) \implies (1)$\\
Vzemimo tako podzaporedje  $\{n_k\}$, da $\sum_{k = 1}^{\infty}{(n_{k+1} - n_k)a_{n_k}}$ konvergira.
Konvergenco vrste $\sum_{n = 1}^{\infty}{a_n}$ bomo dokazali s pomočjo zaporedja delnih vsot.
Označimo z $\{s_m\}$ zaporedje delnih vsot vrste $\sum_{n = 1}^{\infty}{a_n}$.\\
Tedaj za $\forall m \in \N, n_1 < m$, velja sledeča ocena:

\[
    \begin{split}
        0 & < s_m := a_1 + a_2 + \cdots + a_m \\
        & \leq (a_1 + \ldots + a_{n_1}) + (a_{n_{1} + 1} + \ldots + a_{n_{2}})
        + \ldots  + (a_{n_{m} + 1} + \ldots + a_{n_{m+1}})\\
        & \leq n_{1}a_{1} + (n_2 - n_1)a_{n_{1} + 1} + \ldots + (n_{m+1} - n_m)a_{n_{m}+1}\\
        & \leq n_{1}a_{1} + (n_2 - n_1)a_{n_1} + \ldots + (n_{m+1} - n_{m})a_{n_{m}}\\
        & = n_{1}a_{1} + \sum_{k = 1}^{m}{(n_{k+1} - n_k)a_{n_k}} \text{,}
    \end{split}
\]
kjer smo v drugi vrstici upoštevali, da za vsak $m \in \N$ velja $m \leq n_m$, 
torej velja tudi $m \leq n_{m + 1}$; v tretji in četrti vrstici pa smo upoštevali predpostavko,
da je zaporedje $a_n$ padajoče.

Iz ocene sledi, da je zaporedje ${s_n}$ omejeno, saj vrsta $\sum_{k = 1}^{\infty}{(n_{k+1} - n_k)a_{n_k}}$ konvergira, 
kar implicira, da je zaporedje njenih delnih vsot $\sum_{k = 1}^{m}{(n_{k+1} - n_k)a_{n_k}}$ omejeno.
Zaporedje ${s_n}$ je torej omejeno, očitno je tudi monotono, od tod sledi njegova konvergenca.
S tem je izrek dokazan.
\qed

\begin{posledica}(O divergenci).
    Naj bo $a_n$ padajoče zaporedje nenegativnih števil in $ q \in {\mathbb{N}}$, $q > 1$. Če velja:
    \[
        \lim_{n \to \infty}{q^ka_{q^k}} \neq 0 \text{,} 
    \]
    tedaj $\sum_{n = 1}^{\infty}{a_n}$ divergira.
\end{posledica}

\noindent
{\em Dokaz:\/}
Potreben pogoj za konvergenco vrste $\sum_{n = 1}^{\infty}{a_n}$ je
$\lim_{n \to \infty}{a_n} = 0$. Ta pogoj preverimo za vrsto $\sum_{k = 1}^{\infty}{q^ka_{q^k}}$,
ki po implikaciji iz 1. v 2. iz izreka 4 konvergira natanko tedaj, ko konvergira vrsta $\sum_{n = 1}^{\infty}{a_n}$.
\qed

\begin{zgled}
    Ponovno vzemimo harmonično vrsto $\sum_{n = 1}^{\infty}{\frac{1}{n}}$. \\
    Hitro lahko opazimo, da je v tem primeru zaporedje $q^ka_{q^k}$ konstantno enako 1, torej v limiti ne gre proti 0.
\end{zgled}


\begin{opomba}
    Pri študiju konvergence vrst $\sum_{n = 1}^{\infty}{a_n}$ z nenegativnimi členi, 
    za padajoče zaporedje $\{a_n\}$ se pogosto omenja \emph{Cauchyjev kondenzacijski kriterij}:
    \[
        \sum_{n = 1}^{\infty}{a_n} \quad \text{konvergira} \iff \text{konvergira vrsta}
        \sum_{n = 1}^{\infty}{2^ka_{2^k}}
    \]
    Izrek 4 je posplošitev tega kriterija za $q = 2$.
\end{opomba}

\noindent
Poglejmo si uporabo kriterija na zgledu.
\begin{zgled}
    Vzemimo vrsto $\sum_{n = 1}^{\infty}{\frac{1}{nln(n)}}$, za katero kvocientni kriterij odpove.
    \[
        \sum_{k = 1}^{\infty}{q^k\frac{1}{q^kln(q^k)}} =
        \sum_{k = 1}^{\infty}{\frac{1}{kln(q))}}
    \]
    Za vrsto na desni vemo, da divergira. Sledi, da vrsta $\sum_{n = 1}^{\infty}{\frac{1}{nln(n)}}$ divergira.
\end{zgled}


\section{Posplošeni kvocientni kriterij}

Zapišimo osrednji izrek tega besedila.

\begin{izrek}\textbf{(Poslošeni kvocientni kriterij})
    Naj bo $q \in \N$, $q > 1$ in $\{a_n\}$ zaporedje,
    ki zadošča $a_{n} \geq a_{n+1} \geq 0$, za vsak $n \in \N$.
    Tedaj velja:
    
    \begin{enumerate}
        \item Če $\lim_{k \to \infty}{\frac{a_{q^{k + 1}}}{a_{q^k}}} < \frac{1}{q}$,
        potem vrsta $\sum_{n = 1}^{\infty}{a_n}$ konvergira.
        \item Če $\lim_{k \to \infty}{\frac{a_{q^{k + 1}}}{a_{q^k}}} > \frac{1}{q}$,
        potem vrsta $\sum_{n = 1}^{\infty}{a_n}$ divergira.
    \end{enumerate}
\end{izrek}

\noindent
Izrek bomo dokazali s pomočjo izreka 4 in običajnega kvocientnega kriterija.\\
\noindent
{\em Dokaz:\/} Naj bo $\{a_n\}$ padajoče zaporedje nenegativnih števil in $ q \in {\mathbb{N}}$, $q > 1$.
Po izreku 4, vemo, da vrsta $\sum_{n = 1}^{\infty}{a_n}$ konvergira natanko tedaj, ko konvergira vrsta
$\sum_{k = 1}^{\infty}{q^ka_{q^k}}$, $q \in {\mathbb{N}}$, $q > 1$.

Želeli bi si uporabiti kvocienti kriterij za vrsto $\sum_{k = 1}^{\infty}{q^ka_{q^k}}$.
Denimo, da obstaja limita:
\[ 
    \lim_{k \to \infty}{\frac{q^{k + 1}a_{q^{k + 1}}}{q^ka_{q^k}}}  =
    \lim_{k \to \infty}{\frac{qa_{q^{k + 1}}}{a_{q^k}}}  
    \text{.}
\]
Tedaj velja:
\begin{itemize}
    \item Če je $\lim_{k \to \infty}{\frac{qa_{q^{k + 1}}}{a_{q^k}}} < 1$, vrsta $\sum_{k = 1}^{\infty}{q^ka_{q^k}}$ konvergira.\\
    To lahko zapišemo tudi kot:\\
    Če je $\lim_{k \to \infty}{\frac{a_{q^{k + 1}}}{a_{q^k}}} < \frac{1}{q}$, vrsta $\sum_{k = 1}^{\infty}{q^ka_{q^k}}$ konvergira.
    \item Za drugo točko uporabimo enak razmislek.\\
    Če $\lim_{k \to \infty}{\frac{a_{q^{k + 1}}}{a_{q^k}}} > \frac{1}{q}$
    vrsta $\sum_{k = 1}^{\infty}{q^ka_{q^k}}$ divergira. 
\end{itemize}
\qed


\begin{trditev}
    Naj bo $\{a_{n}\}$ padajoče zaporedje in naj obstaja limita
    $D := \lim_{n \to \infty}{\frac{a_{n + 1}}{a_n}} < 1$. Tedaj za $q \in {\mathbb{N}}$, $q > 1$ velja:
    \[
        \lim_{k \to \infty}{\frac{a_{q^{k + 1}}}{a_{q^k}}} = 0
    \]
\end{trditev}

TODO: DOKAZ TRDITVE
Dokaz trditve je preprost, vendar ga bomo izpustili. Najdete ga lahko v \cite{convergence}.

\begin{opomba}
    Trditev pove, da bo na primerih, kjer lahko učinkovito uporabimo kvocientni kriterij,
    gotovo deloval tudi posplošeni kvocientni kriterij. Obratno ne velja, kar vidimo v sledečih primerih.
\end{opomba}

TODO: PRIMER Z LOGARITMI
\noindent
Za konec se vrnimo k začetnemu zgledu.
\begin{zgled}
    Vzemimo vrsto $\sum_{n = 1}^{\infty}{\frac{1}{n^p}}$, $p \in \N$, $p > 0$ pri kateri
    že vemo, da kvocientni kriterij odpove. Poskusimo določiti konvergenco s posplošenim kvocientnim kriterijem.
    Naj bo $q \in \N$, $q > 1$, $a_n = \frac{1}{n^p}$.
    \[
            \lim_{k \to \infty}{\frac{a_{q^{k + 1}}}{a_{q^k}}} =
            \lim_{k \to \infty}{(\frac{q^k}{q^{k+1}})^p} = \frac{1}{q^p}
    \]
    Po izreku 5 sledi konvergenca za $p>1$ in divergenca za $ 0< p<1$.
\end{zgled}

\section{Angleško-slovenski slovar strokovnih izrazov}

\geslo{ratio test}{kvocientni kriterij}

\geslo{convergence (of a series)}{konvergenca (vrste)}

\geslo{decreasing sequence}{padajoče zaporedje}

\geslo{absolutely convergent (series)}{absolutno konvergenta (vrsta)}

\geslo{subsequence}{podzaporedje}

\geslo{harmonic series}{harmonična vrsta}

\geslo{bounded (sequence)}{omejeno (zaporedje)}

\geslo{necessary condition}{potreben pogoj}

\geslo{Cauchy condensation test}{Cauchyjev kondenzacijski test/kriterij}





%%%%%%%%%%%%%%%%%%%%%%%%%%%%%%%%%%%%%%%%%%%%%%%%%%%%55






%%%%%%%%%%%%%%%%%%%%%%%%%%%%%%%%%%%%%%%%%%%%%%%%%%%%%%%%%%%%%%%%%%%%%%%%%%%%%%
%%%%%%%%%%%%%%%%%%%%%%%%%%%%%%%%%%%%%%%%%%%%%%%%%%%%%%%%%%%%%%%%%%%%%%%%%%%%%%





\bibliographystyle{siam}
\bibliographystyle{magic}

\begin{thebibliography}{1}
    \bibitem{convergence}
    Z. Wu; Two convergence theorems and an extension of the ratio test for a
    series. Math. Mag. 92 (2019), no. 3, 222–227
\end{thebibliography}

%%%%%%%%%%%%%%%%%%%%%%%%%%%%%%%%%%%%%%%%%%%%%%

\end{document}