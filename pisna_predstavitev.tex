\documentclass[a4paper,12pt]{article}

\usepackage[slovene]{babel}
\usepackage{amsfonts,amssymb,amsmath}
\usepackage[utf8]{inputenc}
\usepackage[T1]{fontenc}
\usepackage{lmodern}



\def\N{\mathbb{N}} % mnozica naravnih stevil
\def\Z{\mathbb{Z}} % mnozica celih stevil
\def\Q{\mathbb{Q}} % mnozica racionalnih stevil
\def\R{\mathbb{R}} % mnozica realnih stevil
\def\C{\mathbb{C}} % mnozica kompleksnih stevil
\newcommand{\geslo}[2]{\noindent\textbf{#1} \quad \hangindent=1cm #2\\[-1pc]}


\def\qed{$\hfill\Box$}   % konec dokaza
\def\qedm{\qquad\Box}   % konec dokaza v matematičnem načinu
\newtheorem{izrek}{Izrek}
\newtheorem{trditev}{Trditev}
\newtheorem{posledica}{Posledica}
\newtheorem{lema}{Lema}
\newtheorem{opomba}{Opomba}
\newtheorem{definicija}{Definicija}
\newtheorem{zgled}{Zgled}

\title{Posplošitev kvocientnega kriterija za konvergenco vrste \\ 
\Large Seminar}
\author{Timotej Stibilj \\
Fakulteta za matematiko in fiziko \\
Oddelek za matematiko}
\date{21.\ marec 2021}

\begin{document}

\maketitle

\section{uvod}

\emph{Kvocientni kriterij}, imenovan tudi \emph{D'Alembertov kriterij} po francoskem matematiku Jeanu le Rond d'Alembertu
nam je pogosto v pomoč pri določevanju konvergence vrst $\sum_{n = 1}^{\infty}{a_n}$ z nenegativnimi členi. 
V poštev pride predvsem pri vrstah, ki vsebujejo n-te potence števil, npr. $\sum_{n = 1}^{\infty}{\frac{a^n}{n^s}}$, 
in pri vrstah, ki vsebujejo člen n faktorsko, npr. $\sum_{n = 1}^{\infty}{\frac{a^n}{n!}}$.
Spomnimo se, kako ga uporabimo:\\

\noindent
\begin{izrek}(Kvocientni kriterij)\\
    Če obstaja limita:
    \[
        D := \lim_{n \to \infty} \frac{a_{n + 1}}{a_n} \text{,}
    \]
    potem velja:
    \begin{itemize}
        \item če je $ D < 1 $ vrsta konvergira,
        \item če je $D > 1$ vrsta divergira,
        \item za $D = 1$ kriterij odpove.
    \end{itemize}
\end{izrek}

Poglejmo si preprost primer, pri katerem kvocientni kriterij odpove.

\begin{zgled}
    Vzemimo $a_n$ = $\frac{1}{n^p}$, $p \in \N$.
    \[
        D = \lim_{n \to \infty} \frac{a_{n + 1}}{a_n}
        = \lim_{n \to \infty} \frac{\frac{1}{(n+1)^p}}{\frac{1}{n^p}}
        = \lim_{n \to \infty} (\frac{n}{n + 1})^p
        = 1
    \]
    Limita $ \lim_{n \to \infty} \frac{a_{n + 1}}{a_n} $ je torej 1, od koder sledi, da si s kvocientnim kriterijem ne moremo pomagati.
\end{zgled}

Naravno se je vprašati, kako lahko kvocientni kriterij posplošimo, da bo zadostoval za določanje konvergence,
kjer kvocienti kriterij odpove, obenem tudi za prejšnji primer. S tem v posebnem dobimo tudi alternativni
dokaz za divergenco harmonične vrste.\\
Možnih razširitev kriterija je več. V tem besedilu se bomo osredotočili na preproste kriterije, ki ne
zahtevajo poznavanja zahtevnih matematičnih konceptov. Predstavili bomo vrsto ekvivalentnih trditev
za konvergenco vrste $\sum_{n = 1}^{\infty}{a_n}$, kjer je $\{a_n\}$ padajoče zaporedje nenegativnih realnih števil, s preprostimi dokazi. Osrednja
izreka, ki ju želimo predstaviti sta:

\begin{itemize}
    \item Za $ 1 < q \in \N$, $\sum_{n = 1}^{\infty}{q^na_{q^n}}$ konvergira $\iff \sum_{n = 1}^{\infty}{a_{n}}$ konvergira.
    \item Posplošeni kvocientni kriterij, ki ga dobimo s pomočjo zgornjega izreka in običajnega kvocientnega kriterija.
    Le-tega lahko seveda, kakor ime pove, uporabimo v nekaterih primerih, v katerih običajen kvocientni kriterij odpove, 
    hkrati pa njegova uporaba, v smislu težavnosti računanja, ostaja enostavna.
\end{itemize}


Vsi izreki, ki so predstavljeni v nadaljevanju, veljajo za padajoča zaporedja nenegativnih števil.
Kaj lahko naredimo v primeru, ko imamo opravka z zaporedje $a_n$, ki ni padajoče?
Spomnimo se na izrek o preureditvi absolutno konvergentne vrste.

\begin{izrek}
    Naj bo vrsta $\sum_{n = 1}^{\infty}{a_n}$ absolutno konvergentna.\\
    Tedaj za vsako bijekcijo:
    \[\pi: \N \rightarrow \N \]
    tudi vrsta $\sum_{n = 1}^{\infty}{a_{\pi(n)}}$ konvergira in vsota vrst je enaka.
\end{izrek}

Torej lahko v primeru, ko se ukvarjamo z vsoto vrste $\sum_{n = 1}^{\infty}$, kjer je $\{a_n\}$ zaporedje z nenegativnimi členi, BŠS predpostavimo, 
da je $\{a_n\}$ padajoče. 

\section{Dva izreka o konvergenci vrst}
Pri določevanju konvergence vrste $\sum_{n = 1}^{\infty}{a_n}$, si lahko pomagamo s konvergenco vrste
$\sum_{k = 1}^{\infty}{(n_{k+1} - n_k)a_{n_{k}}}$, kjer je $\{n_k\}$ podzaporedje zaporedja $\{n\}$ naravnih števil.
Formulirajmo izrek.

\begin{izrek}
    Naj bo $\{a_n\}$ padajoče zaporedje nenegativnih realnih števil. \\
    
    Tedaj so ekvivalentne naslednje izjave:
    \begin{enumerate}
        \item $\sum_{n = 1}^{\infty}{a_n}$ konvergira
        \item za vsako podzaporedje $\{n_k\}$ zaporedja $\{n\}$ \\
        konvergira $\sum_{k = 1}^{\infty}{(n_{k+1} - n_k)a_{n_{k + 1}}}$
        \item $\forall M > 0$, če za podzaporedje $\{n_k\}$ zaporedja $\{a_n\}$ velja:
        \[
            (n_{k+1} - n_k) \leq M(n_k - n_{k - 1}),  \forall \quad 1 < k \in \N 
        \]
        potem $\sum_{k = 1}^{\infty}{(n_{k+1} - n_k)a_{n_{k}}}$ konvergira.
        \item $\forall M > 0$ $\exists$ podzaporedje $\{n_k\}$ zaporedja $\{n\}$, tako da velja:
        \[
            1 < n_{k+1} - n_k < (M+1)(n_k - n_{k - 1}),  \forall \quad 1 < k \in \N 
        \]
        in $\sum_{k = 1}^{\infty}{(n_{k+1} - n_k)a_{n_{k}}}$ konvergira.
    \end{enumerate}
\end{izrek}

\noindent
Dokaz tega izreka bomo izpustili. Najdete ga v \cite{convergence}.

\begin{zgled}(Alternativni dokaz divergence harmonične vrste)\\
    Denimo, da imamo podzaporedje $\{n_k\}$ zaporedja $\{n\}$, 
    za katerega $\sum_{k = 1}^{\infty}{(n_{k+1} - n_k)a_{n_{k + 1}}}$ divergira. Potem po implikaciji
    $ (1) \implies (2)$ vrsta $\sum_{n = 1}^{\infty}{a_n}$ divergira.\\
    Za harmonično vrsto $\sum_{n = 1}^{\infty}{\frac{1}{n}}$ vzemimo $\{n_k\} = k!$. Sledi:
    \[
        \sum_{k = 1}^{\infty}{(n_{k+1} - n_k)a_{n_{k + 1}}} =
        \sum_{k = 1}^{\infty}{((k + 1)! - k!) \frac{1}{(k+1)!}}=
        \sum_{k = 1}^{\infty}{\frac{k}{k + 1}}
    \]
    Vrsta $\sum_{k = 1}^{\infty}{\frac{k}{k + 1}}$ divergira,
    saj $\lim_{k \to \infty}{\frac{k}{k + 1}} \neq 0$. Sledi divergenca harmonične vrste.
\end{zgled}


Iz ekvivalence $(1) \iff (2)$ iz izreka 3 in potrebnega pogoja za konvergenco vrste sledi naslednja posledica.
\begin{posledica}
    Naj bo ${a_n}$ padajoče zaporedje nenegativnih realnih števil.
    Če obstaja podzaporedje $\{n_k\}$ zaporedja \{n\} naravnih števil, da velja:
    \[
        \lim_{k \to \infty} (n_{k+1} - n_k)a_{n_{k+1}} \neq 0 
    \] 
    potem $\sum_{n = 1}^{\infty}{a_n}$ divergira.
\end{posledica}

Po izreku 3 velja: 
Konvergenca vrste $\sum_{n = 1}^{\infty}{a_n}$ implicira obstoj takega
podzaporedja $\{n_k\}$ od $\{n\}$, da $\sum_{k = 1}^{\infty}{(n_{k+1} - n_k)a_{n_{k + 1}}}$ konvergira.
Obratno, če $\sum_{k = 1}^{\infty}{(n_{k+1} - n_k)a_{n_{k + 1}}}$, sledi da 
$\sum_{n = 1}^{\infty}{a_n}$ konvergira (glej dokaz $(4) \implies (1)$).
Izkaže se, da lahko konvergenco $\sum_{n = 1}^{\infty}{a_n}$ popolnoma karakteriziramo
z $\sum_{k = 1}^{\infty}{q^ka_{q^k}}$, torej če za $\{n_k\}$ izberemo ${q^k}$, kjer je $q \in \N$, $q > 1$.

\begin{izrek}
    Naj bo $\{a_n\}$ padajoče zaporedje in $ q \in {\mathbb{N}}$, $q > 1$. Tedaj je ekvivalentno:
    \begin{enumerate}
        \item $\sum_{n = 1}^{\infty}{a_n}$ konvergira,
        \item $\sum_{k = 1}^{\infty}{q^ka_{q^k}}$ konvergira,
        \item $\exists$ podzaporedje $\{n_k\}$ zaporedja $\{n\}$, da je $\{n_{k+1} - n_k\}$ 
        naraščajoče in konvergirata vrsti:\\
        $\sum_{k = 1}^{\infty}{\frac{1}{n_{k+1} - n_k}}$ in 
        $\sum_{k = 1}^{\infty}{(n_{k+1} - n_k)a_{n_k}}$  
        \item $\exists$ podzaporedje $\{n_k\}$ od $\{n\}$, da $\sum_{k = 1}^{\infty}{(n_{k+1} - n_k)a_{n_k}}$ konvergira.
    \end{enumerate}
\end{izrek}   

\noindent
Dokazali bomo verigo implikacij. $ (1)\implies (2)\implies (3) \implies (4)\implies(1)$.
\noindent
\textbf{Dokaz}:\\
$(1) \implies (2)$\\
Naj bo $ 1 < q \in \N$ in naj $\sum_{n = 1}^{\infty}{a_n}$ konvergira.\\
    Uporabimo implikacijo $(1) \implies (3)$ iz izreka 3, za $n_k = q^k$:\\
    \[
        n_{k+1} - n_k \leq q(q^k - q^{k-1}) = q(n_{k} - n_{k - 1})
    \]
    Zaporedje $\{n_k\}$ torej zadošča pogoju $(n_{k+1} - n_k) \leq M(n_k - n_{k - 1})$,
    $\forall 1 < k \in \N$ za $M = q$.
    Sledi:
    \[
        \sum_{k = 1}^{\infty}{(n_{k+1} - n_k)a_{n_k}} = (q-1)\sum_{k = 1}^{\infty}{q^ka_{q^k}} 
        \quad \text{konvergira.}
    \]
    Od tod pa očitno sledi, da $\sum_{k = 1}^{\infty}{q^ka_{q^k}}$ konvergira.

\noindent
$(2) \implies (3)$\\

Vzemimo ${n_k} = q^k$.
\begin{itemize}
    \item $\{n_{k+1} - n_k\}$ je naraščajoče, saj $ q > 1$,
    \item
    $\sum_{k = 1}^{\infty}{\frac{1}{n_{k+1} - n_k}} = 
    \sum_{k = 1}^{\infty}{\frac{1}{q^{k + 1} - q^k}} =
    \frac{1}{q - 1}\sum_{k = 1}^{\infty}{(\frac{1}{q})^k}$
    To je geometrijska vrsta za $\text{koef.} = \frac{1}{q}$ za $q > 1$. Torej je konvergentna.
    \item 
    $\sum_{k = 1}^{\infty}{(n_{k+1} - n_k)a_{n_k}} =
    \sum_{k = 1}^{\infty}{(q^{k + 1} - q^k)}a_{n_k} =
    (q - 1)\sum_{k = 1}^{\infty}{q^ka_{q^k}}$
    Iz predpostavke, da $\sum_{k = 1}^{\infty}{q^ka_{q^k}}$ konvergira, torej sledi,
    da $\sum_{k = 1}^{\infty}{(n_{k+1} - n_k)a_{n_k}}$ konvergira.
\end{itemize}

\noindent
$(3) \implies (4)$\\

Implikacija je očitna. Za $\{n_k\}$ vzamemo isto zaporedje kot v točki 3.
Ostaja nam še implikacija $(4) \implies (1)$.

\noindent
$(4) \implies (1)$\\
Vzemimo tako podzaporedje  $\{n_k\}$, da $\sum_{k = 1}^{\infty}{(n_{k+1} - n_k)a_{n_k}}$ konvergira.
Konvergenco vrste $\sum_{n = 1}^{\infty}{a_n}$ bomo dokazali s pomočjo zaporedja delnih vsot.
Označimo z $\{s_m\}$ zaporedje delnih vsot vrste $\sum_{n = 1}^{\infty}{a_n}$.\\
Tedaj za $\forall m \in \N, n_1 < m$, velja:
\[
    0 < s_m := a_1 + a_2 + \cdots + a_m \leq n_{1}a_{1} + \sum_{k = 1}^{m}{(n_{k+1} - n_k)a_{n_k}}
\]
Od tod sledi, da je zaporedje ${s_n}$ omejeno, saj vrsta $\sum_{k = 1}^{\infty}{(n_{k+1} - n_k)a_{n_k}}$ konvergira, 
kar implicira, da je zaporedje njenih delnih vsot $\sum_{k = 1}^{m}{(n_{k+1} - n_k)a_{n_k}}$ omejeno.
Zaporedje ${s_n}$ je torej omejeno, očitno je tudi monotono, od tod sledi njegova konvergenca.
S tem je izrek dokazan.
\qed

\begin{posledica}(O divergenci)\\
    Naj bo $a_n$ padajoče zaporedje. Če velja:
    \[
        \lim_{n \to \infty}{q^ka_{q^k}} \neq 0 \text{,} 
    \]
    tedaj $\sum_{n = 1}^{\infty}{a_n}$ divergira.
\end{posledica}

\noindent
\textbf{Dokaz}:\\
Sledi direktno iz potrebnega pogoja konvergenco vrste $\sum_{n = 1}^{\infty}{a_n}$,
$\lim_{n \to \infty}{a_n} = 0$ in implikacije $ (1) \implies (2)$.
\qed

\begin{opomba}
    Pri študiju konvergence vrst $\sum_{n = 1}^{\infty}{a_n}$ z nenegativnimi členi, 
    za padajoče zaporedje $\{a_n\}$ se pogosto omenja \emph{Cauchyjev kondenzacijski kriterij}:
    \[
        \sum_{n = 1}^{\infty}{a_n} \quad \text{konvergira} \iff \text{konvergira vrsta}
        \sum_{n = 1}^{\infty}{2^ka_{2^k}}
    \]
    Naš izrek je posplošitev tega kriterija za $q = 2$.
\end{opomba}

\noindent
Poglejmo si uporabo kriterija na zgledu.
\begin{zgled}
    Vzemimo vrsto $\sum_{n = 1}^{\infty}{\frac{1}{nln(n)}}$, za katero kvocientni kriterij odpove.
    \[
        \sum_{k = 1}^{\infty}{q^k\frac{1}{q^kln(q^k)}} =
        \sum_{k = 1}^{\infty}{\frac{1}{kln(q))}}
    \]
    Za vrsto na desni vemo, da divergira. Sledi divergenca $\sum_{n = 1}^{\infty}{\frac{1}{nln(n)}}$.
\end{zgled}


\section{Posplošeni kvocientni kriterij}
Po izreku 4, vemo: \\
$\sum_{n = 1}^{\infty}{a_n}$ konvergira $\iff$
$\sum_{k = 1}^{\infty}{q^ka_{q^k}}$, $q \in {\mathbb{N}}$, $q > 1$  konvergira.

Zapišimo kvocienti kriterij za $\sum_{k = 1}^{\infty}{q^ka_{q^k}}$. Ta vrsta konvergira, če velja:

\[  
    D := 
    \lim_{k \to \infty}{\frac{q^{k + 1}a_{q^{k + 1}}}{q^ka^k}}  =
    \lim_{k \to \infty}{\frac{qa_{q^{k + 1}}}{a^k}} < 1
    \iff
    \lim_{k \to \infty}{\frac{a_{q^{k + 1}}}{a^k}} < \frac{1}{q}
\]

\noindent
Seveda velja tudi:\\
če $\lim_{k \to \infty}{\frac{a_{q^{k + 1}}}{a^k}} > \frac{1}{q}$
vrsta $\sum_{k = 1}^{\infty}{q^ka_{q^k}}$ divergira. 


Od tod sledi osrednji izrek tega besedila, t. j. posplošeni kvocientni kriterij.

\begin{izrek}\textbf{(Poslošeni kvocientni kriterij})\\
Naj bo $q \in \N$, $q > 1$ in ${a_n}$ zaporedje,
ki zadošča $a_{n} \geq a_{n+1} \geq 0 \text{,} \quad \forall n \in \N$.
Tedaj velja:

\begin{enumerate}
    \item Če $\lim_{k \to \infty}{\frac{a_{q^{k + 1}}}{a^k}} < \frac{1}{q}$,
    potem vrsta $\sum_{n = 1}^{\infty}{a_n}$ konvergira.
    \item Če $\lim_{k \to \infty}{\frac{a_{q^{k + 1}}}{a^k}} > \frac{1}{q}$,
    potem vrsta $\sum_{n = 1}^{\infty}{a_n}$ divergira.
\end{enumerate}
\end{izrek}

\noindent
\textbf{Dokaz}:\\
Uporabimo izrek 4 in običajni kvocientni kriterij.
\qed

\begin{trditev}
    Naj bo $a_{n}$ padajoče zaporedje in naj obstaja limita
    $D := \lim_{n \to \infty}{\frac{a_{n + 1}}{a_n}}$. Tedaj za $q \in {\mathbb{N}}$, $q > 1$ velja:
    \[
        \lim_{k \to \infty}{\frac{a_{q^{k + 1}}}{a_{q^k}}} = 0
    \]
\end{trditev}

Dokaz trditve je preprost, vendar ga bomo izpustili. Najdete ga lahko v \cite{convergence}.

\begin{opomba}
    Trditev pove, da lahko posplošeni kvocientni kriterij uporabimo za dokaz običajnega
    kvocientnega kriterija. Na primerih, kjer lahko učinkovito uporabimo kvocientni kriterij,
    bo gotovo deloval tudi posplošeni kvocientni kriterij. Obratno seveda ne velja.
\end{opomba}

\noindent
Za konec se vrnimo k začetnemu zgledu.
\begin{zgled}
    Vzemimo vrsto $\sum_{n = 1}^{\infty}{\frac{1}{n^p}}$, $p \in \N$, $p > 0$ pri kateri
    že vemo, da kvocientni kriterij odpove. Poskusimo določiti konvergenco s posplošenim kvocientnim kriterijem.
    Naj bo $q \in \N$, $q > 1$, $a_n = \frac{1}{n^p}$.
    \[
            \lim_{k \to \infty}{\frac{a_{q^{k + 1}}}{a_{q^k}}} =
            \lim_{k \to \infty}{(\frac{q^k}{q^{k+1}})^p} = \frac{1}{q^p}
    \]
    Po izreku 5 sledi konvergenca za $p>1$ in divergenca za $ 0< p<1$.
\end{zgled}

\section{Angleško-slovenski slovar strokovnih izrazov}

\geslo{ratio test}{kvocientni kriterij}

\geslo{convergence (of a series)}{konvergenca (vrste)}

\geslo{decreasing sequence}{padajoče zaporedje}

\geslo{absolutely convergent (series)}{absolutno konvergenta (vrsta)}

\geslo{subsequence}{podzaporedje}

\geslo{harmonic series}{harmonična vrsta}

\geslo{bounded (sequence)}{omejeno (zaporedje)}

\geslo{necessary condition}{potreben pogoj}

\geslo{Cauchy condensation test}{Cauchyjev kondenzacijski test/kriterij}





%%%%%%%%%%%%%%%%%%%%%%%%%%%%%%%%%%%%%%%%%%%%%%%%%%%%55

\begin{abstract}
\end{abstract}


\tableofcontents

\pagebreak


%%%%%%%%%%%%%%%%%%%%%%%%%%%%%%%%%%%%%%%%%%%%%%%%%%%%%%%%%%%%%%%%%%%%%%%%%%%%%%
%%%%%%%%%%%%%%%%%%%%%%%%%%%%%%%%%%%%%%%%%%%%%%%%%%%%%%%%%%%%%%%%%%%%%%%%%%%%%%





\bibliographystyle{siam}
\bibliographystyle{magic}

\begin{thebibliography}{1}
    \bibitem{convergence}
    Z. Wu; Two convergence theorems and an extension of the ratio test for a
    series. Math. Mag. 92 (2019), no. 3, 222–227
\end{thebibliography}

%%%%%%%%%%%%%%%%%%%%%%%%%%%%%%%%%%%%%%%%%%%%%%

\end{document}